\chapter*{Введение}                         % Заголовок
\addcontentsline{toc}{chapter}{Введение}    % Добавляем его в оглавление

\newcommand{\actuality}{}
\newcommand{\progress}{}
\newcommand{\aim}{{\textbf\aimTXT}}
\newcommand{\tasks}{\textbf{\tasksTXT}}
\newcommand{\novelty}{\textbf{\noveltyTXT}}
\newcommand{\influence}{\textbf{\influenceTXT}}
\newcommand{\methods}{\textbf{\methodsTXT}}
\newcommand{\defpositions}{\textbf{\defpositionsTXT}}
\newcommand{\reliability}{\textbf{\reliabilityTXT}}
\newcommand{\probation}{\textbf{\probationTXT}}
\newcommand{\contribution}{\textbf{\contributionTXT}}
\newcommand{\publications}{\textbf{\publicationsTXT}}

Ассоциативное кольцо $A$ всегда превращается в кольцо Ли $A^{(-)}$, если
умножение в $A$ заменим новым $[a,b]\coloneq ab - ba$ (коммутирование).
А.\:А.~Альберт \cite{albert48} называет произвольную алгебру $A$ (не
обязательно ассоциативную) \emph{Ли-допустимой}, когда $A^{(-)}$ есть алгебра
Ли; см. также \cite{myung72}. Согласно \cite{DAN2018} и ~\cite{imm2020}, $A$
называется \emph{точной обертывающей} алгебры Ли $L$, если $A^{(-)}\simeq L$.

\medskip

Алгебру Шевалле над полем $K$ характеризуют системой корней $\Phi$ и базой,
состоящей из элементов $e_r$ $(r \in \Phi)$ и подходящей базы подалгебры
Картана. Как показано в \cite[\S\ 4.2]{carter72}, структурные константы базы
Шевалле однозначно определяет их выбор для \emph{нильтреугольной} подалгебры
$N\Phi(K)$ с базой $\{e_r\ | \ r \in \Phi^+ \}$. Унипотентная подгруппа
$U\Phi(K)$ группы Шевалле типа $\Phi$ над $K$ представлена в \cite{vl90a_ru}
присоединенной группой на $N\Phi(K)$. Как правило, ее нормальные подгруппы --
это, в точности, идеалы кольца Ли $N\Phi(K)$, \cite{LevSul-DAN, JA-12}.

Для корней считаем $s\geq r$, когда в разложении $s-r$ по базе $\Pi$ в $\Phi^+$
все коэффициенты неотрицательны. Корни $r$ и $s$ в $\Phi^+$ назовем
\emph{инцидентными}, если $s \geq r$ или $r\geq s$. Любое множество
$\mathcal{L}$ попарно неинцидентных корней в $\Phi^+$ называем \emph{множеством
    углов в $\Phi^+$}. Выделим в $N\Phi(K)$ идеалы
%
\[T(r)=\sum_{s \geq r}Ke_s,\quad Q(r)=\sum_{s>r}Ke_s\quad (r\in
    \Phi^+),\quad Q(\mathcal{L})= \sum_{r \in \mathcal{L}}Q(r).\]
%
Если $H \subseteq T(\mathcal{L})\coloneq\sum_{r \in \mathcal{L}} T(r)$ и включение
нарушается при любой замене $T(r)$ на $Q(r)$ в сумме, то множество
$\mathcal{L}= \mathcal{L}(H)$ определено однозначно и, согласно \cite[п.
    3]{DAN2018}, называется \emph{множеством углов в $H$}.

Идеал $H$ кольца Ли $N\Phi(K)$ называем \emph{стандартным}, если $Q
    (\mathcal{L}(H))\subset H$.

\medskip

Известные взаимосвязанные перечисления идеалов колец Ли $N\Phi(K)$ и нормальных
подгрупп групп $U\Phi(K)$ редуцировались к перечислениям идеалов вида
$T(\mathcal{L})$ (\cite[следствие 4.3]{vL74}, \cite[теорема
    2.1.2]{Egorychev84}, \cite{Eg-Lev-1996, eS05} и др.) и, таким образом, к
перечислениям путей в различных решетках.

Для классических типов в 2001 году в~\cite{sigsam2001} записана, как
проблема~1, задача

\medskip

\noindent\textbf{(A)} \emph{Найти число стандартных идеалов алгебры Ли
    $N\Phi(K)$ над конечным полем $K$.}

\medskip

В \S~\ref{sec:Problem-1} проблему~1 из \cite{sigsam2001} решает теорема
\ref{th:Enum-St-Id}, анонсированная ранее \cite[Теорема 3]{DAN2018}. Для
исключительных типов задачу~\textbf{(A)} решает теорема \ref{th:exhls}.

\medskip

При фиксированном $\Phi$ для алгебры Ли $N\Phi(K)$ точные обертывающие алгебры
$R_{\Phi}$ и их число выявляет предложение \ref{p:mult-enveloping} в
\S~\ref{sec:enveloping}. Известно, что одну из них для $\Phi$ типа $A_{n-1}$
представляет алгебра $NT(n, K)$ нильтреугольных $n\times n$ матриц над $K$; все
идеалы алгебры и кольца $NT(n, K)$ стандартны \cite{Dub-Perl-1951, vL76}.

Алгебру $R_{\Phi}$ называем стандартной, если все ее идеалы стандартны. В силу
теоремы \ref{th:classical-Exc-RDn} и предложения \ref{p:Exc-Dn-En} (см. также
\cite{DAN2018} и \cite{jsfu2018}), стандартная алгебра $R_{\Phi}$ существует
для всех типов, исключая тип $D_n$ $(n \geq 4)$ и $E_n\ (n=6,7,8)$.

Алгебра $NT(n,K)$ нижних нильтреугольных (т.\:е. с нулями на главной диагонали
и над ней) $n \times n$ матриц над полем $K$, с точностью до изоморфизма,
оказывается единственной ассоциативной обертывающей алгеброй $R_{\Phi}$ типа
$A_{n-1}$ (теорема \ref{th:Stand-An} в \S~\ref{sec:uniqtheorems} и
\cite{imm2020}); более слабым является условие стандартности.

Вопрос об условиях однозначности неассоциативной обертывающей алгебры отмечал
И.\:П.~Шестаков на конференции в 2017 году (см. \cite{modernWorldConf2017}).
Для классических типов этот вопрос исследуется в \S~\ref{sec:uniqtheorems}
(теоремы \ref{th:Stand-An}, \ref{th:Stand-Bn} и замечание
\ref{remark:Stand-Cn}).

\medskip

Лемма~\ref{l:LieIdeal} и теорема~\ref{th:ExIdForm} характеризуют нестандартные
идеалы единственной нестандартной алгебры $RD_n(K)$ специальным набором
параметров.

\medskip

В работе используем стандартные обозначения \cite{carter72}, \cite{nB72}:
$\Phi^+$ "--- система поло\-жительных корней, $\Pi$ "--- ее база, $\rho$ "---
максимальный в $\Phi^+$ корень, $ht(r)$ "--- высота корня $r$. Число Кокстера
$h=h(\Phi)$ системы $\Phi$ равно $ht(\rho)+1$. \nomenclature{$\Phi^+$}{система
    поло\-жительных корней} \nomenclature{$\Pi$}{база системы корней}
\nomenclature{$\rho$}{максимальный в $\Phi^+$ корень}
\nomenclature{$ht(r)$}{высота корня $r$} \nomenclature{h(\Phi)}{Число Кокстера
    системы $\Phi$ (равно $ht(\rho)+1$)}

{\aim} данной работы является исследование записанных в 2001 году проблем
комбинаторного перечисления идеалов алгебры Ли $N\Phi(K)$
\cite[Проблемы 1 и 2]{sigsam2001} и нахождение условия однозначности
ее точной обертывающей алгебры.

{\novelty}
\begin{enumerate}[beginpenalty=10000] % https://tex.stackexchange.com/a/476052/104425
    \item Впервые \ldots
    \item Впервые \ldots
    \item Было выполнено оригинальное исследование \ldots
\end{enumerate}

{\influence} Все основные результаты диссертации являются новыми. Работа носит теоретический характер.

{\methods} Развиваемый Г.\:П.~Егорычевым с 1970-х годов метод интегрального представления
и вычисления комбинаторных сумм (метод коэффициентов) находит приложения в
многочисленных задачах алгебры, комбинаторного анализа и других областей
математики, \cite{Egorychev84, Eg-2009, Leontiev2001, Riedel-2023,
    Eg-2013}. Комбинаторная теорема \ref{th:Enum-Dn}, завершающая перечисление всех
идеалов обертывающих алгебр классических типов, впервые применяется для
вычисления 3-кратной комбинаторной суммы с $q$-биномиальными коэффициентами.
См. также замечание~\ref{remark:problem-2}.

    {\probation} Результаты диссертационной работы докладывались на заседаниях Красноярского
алгебраического семинара (2016--2020\,гг.), на cеминаре им.~Н.\:А. Вавилова
(СПбГУ, г.~Санкт-Петербург, 8 декабря 2017\,г.) и апробировались на следующих
конференциях:
\begin{enumerate}
    \item Международная научная конференция студентов, аспирантов и молодых учёных
          <<Проспект Свободный>>, 2016\,г., г.~Красноярск.
    \item Международная конференция, посвященная 70-летию В.\,М.~Левчука <<Алгебра и
          Логика: Теория и Приложения>>, 2016\,г., г.~Красноярск.
    \item Международная конференция, посвященная 60-летию Института математики
          им.~С.\:Л.~Соболева <<Математика в современном мире>>, 2017\,г.,
          г.~Новосибирск.
    \item Международная алгебраическая конференция, посвященная 110-летию со дня рождения
          профессора А.\:Г.~Куроша, 2018\,г., г.~Москва.
    \item Международная конференция <<Мальцевские чтения>>, 2016 и 2023\,г.,
          г.~Новосибирск.
\end{enumerate}

\begin{refsection}[bl-author]
    % Это refsection=1.
    % Процитированные здесь работы:
    %  * подсчитываются, для автоматического составления фразы "Основные результаты ..."
    %  * попадают в авторскую библиографию, при usefootcite==0 и стиле `\insertbiblioauthor` или `\insertbiblioauthorgrouped`
    %  * нумеруются там в зависимости от порядка команд `\printbibliography` в этом разделе.
    %  * при использовании `\insertbiblioauthorgrouped`, порядок команд `\printbibliography` в нём должен быть тем же (см. biblio/biblatex.tex)
    %
    % Невидимый библиографический список для подсчёта количества публикаций:
    \phantom{\printbibliography[heading=nobibheading, section=1, env=countauthorvak,          keyword=biblioauthorvak]%
        \printbibliography[heading=nobibheading, section=1, env=countauthorwos,          keyword=biblioauthorwos]%
        \printbibliography[heading=nobibheading, section=1, env=countauthorscopus,       keyword=biblioauthorscopus]%
        \printbibliography[heading=nobibheading, section=1, env=countauthorconf,         keyword=biblioauthorconf]%
        \printbibliography[heading=nobibheading, section=1, env=countauthorother,        keyword=biblioauthorother]%
        \printbibliography[heading=nobibheading, section=1, env=countregistered,         keyword=biblioregistered]%
        \printbibliography[heading=nobibheading, section=1, env=countauthorpatent,       keyword=biblioauthorpatent]%
        \printbibliography[heading=nobibheading, section=1, env=countauthorprogram,      keyword=biblioauthorprogram]%
        \printbibliography[heading=nobibheading, section=1, env=countauthor,             keyword=biblioauthor]%
        \printbibliography[heading=nobibheading, section=1, env=countauthorvakscopuswos, filter=vakscopuswos]%
        \printbibliography[heading=nobibheading, section=1, env=countauthorscopuswos,    filter=scopuswos]}%
    %
    \nocite{*}%
    %
    {\publications} Основные результаты по теме диссертации изложены в~\arabic{citeauthor}~печатных изданиях,
    \arabic{citeauthorvak} из которых изданы в журналах, рекомендованных ВАК%
    \ifnum \value{citeauthorscopuswos}>0%
        , \arabic{citeauthorscopuswos} "--- в~периодических научных журналах, индексируемых Web of~Science и Scopus%
    \fi%
    \ifnum \value{citeauthorconf}>0%
        , \arabic{citeauthorconf} "--- в~тезисах докладов.
    \else%
        .
    \fi%
    % К публикациям, в которых излагаются основные научные результаты диссертации на соискание учёной
    % степени, в рецензируемых изданиях приравниваются патенты на изобретения, патенты (свидетельства) на
    % полезную модель, патенты на промышленный образец, патенты на селекционные достижения, свидетельства
    % на программу для электронных вычислительных машин, базу данных, топологию интегральных микросхем,
    % зарегистрированные в установленном порядке.(в ред. Постановления Правительства РФ от 21.04.2016 N 335)
\end{refsection}%
\begin{refsection}[bl-author]
    % Это refsection=2.
    % Процитированные здесь работы:
    %  * попадают в авторскую библиографию, при usefootcite==0 и стиле `\insertbiblioauthorimportant`.
    %  * ни на что не влияют в противном случае
    \nocite{vmj2015}%vak
    \nocite{jsfu2018}%patent
    \nocite{imm2020}%program
    \nocite{smj2023}%other
\end{refsection}%
%
% Всё, что вне этих двух refsection, это refsection=0,
%  * для диссертации - это нормальные ссылки, попадающие в обычную библиографию
%  * для автореферата:
%     * при usefootcite==0, ссылка корректно сработает только для источника из `external.bib`. Для своих работ --- напечатает "[0]" (и даже Warning не вылезет).
%     * при usefootcite==1, ссылка сработает нормально. В авторской библиографии будут только процитированные в refsection=0 работы.
 % Характеристика работы по структуре во введении и в автореферате не отличается (ГОСТ Р 7.0.11, пункты 5.3.1 и 9.2.1), потому её загружаем из одного и того же внешнего файла, предварительно задав форму выделения некоторым параметрам

\textbf{Объем и структура работы.} Диссертация состоит из~введения,
\formbytotal{totalchapter}{глав}{ы}{}{},
заключения и
\formbytotal{totalappendix}{приложен}{ия}{ий}{}.
%% на случай ошибок оставляю исходный кусок на месте, закомментированным
%Полный объём диссертации составляет  \ref*{TotPages}~страницу
%с~\totalfigures{}~рисунками и~\totaltables{}~таблицами. Список литературы
%содержит \total{citenum}~наименований.
%
Полный объём диссертации составляет \formbytotal{TotPages}{страниц}{у}{ы}{},
включая \formbytotal{totalcount@figure}{рисун}{ок}{ка}{ков} и
\formbytotal{totalcount@table}{таблиц}{у}{ы}{}. Список литературы содержит
\formbytotal{citenum}{наименован}{ие}{ия}{ий}.
%%

Автор благодарен научным руководителям профессору Левчуку Владимиру Михайловичу
и профессору Сулеймановой Галине Сафиуллановне за постановку задач и внимание к
работе. Признателен сотрудникам кафедры алгебры и математической логики и
Института математики и фундаментальной информатики СФУ за хорошие условия
работы над диссертацией.
