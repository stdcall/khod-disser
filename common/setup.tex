%%% Режим черновика %%%
\makeatletter
\@ifundefined{c@draft}{
  \newcounter{draft}
  \setcounter{draft}{0}  % 0 --- чистовик (максимальное соблюдение ГОСТ)
                         % 1 --- черновик (отклонения от ГОСТ, но быстрая
                         %       сборка итоговых PDF)
}{}
\makeatother


%%% Библиография %%%
\makeatletter
\@ifundefined{c@bibliosel}{
  \newcounter{bibliosel}
  \setcounter{bibliosel}{1}   % 0 --- встроенная реализация с загрузкой файла
                              %       через движок bibtex8;
                              % 1 --- реализация пакетом biblatex через движок
                              %       biber
}{}
\makeatother

%%% Вывод типов ссылок в библиографии %%%
\makeatletter
\@ifundefined{c@mediadisplay}{
  \newcounter{mediadisplay}
  \setcounter{mediadisplay}{2}   % 0 --- не делать ничего; надписи [Текст] и
                                 %       [Эл. ресурс] будут выводиться только в ссылках с
                                 %       заполненным полем `media`;
                                 % 1 --- автоматически добавлять надпись [Текст] к ссылкам с
                                 %       незаполненным полем `media`; таким образом, у всех
                                 %       источников будет указан тип, что соответствует
                                 %       требованиям ГОСТ
                                 % 2 --- автоматически удалять надписи [Текст], [Эл. Ресурс] и др.;
                                 %       не соответствует ГОСТ
                                 % 3 --- автоматически удалять надпись [Текст];
                                 %       не соответствует ГОСТ
                                 % 4 --- автоматически удалять надпись [Эл. Ресурс];
                                 %       не соответствует ГОСТ
}{}
\makeatother
