\documentclass[a4paper,14pt,oneside,openany]{memoir}
\newif\ifsynopsis  % Условие, проверяющее, что документ --- автореферат
\synopsisfalse

%%% Режим черновика %%%
\makeatletter
\@ifundefined{c@draft}{
  \newcounter{draft}
  \setcounter{draft}{0}  % 0 --- чистовик (максимальное соблюдение ГОСТ)
                         % 1 --- черновик (отклонения от ГОСТ, но быстрая
                         %       сборка итоговых PDF)
}{}
\makeatother


%%% Библиография %%%
\makeatletter
\@ifundefined{c@bibliosel}{
  \newcounter{bibliosel}
  \setcounter{bibliosel}{1}   % 0 --- встроенная реализация с загрузкой файла
                              %       через движок bibtex8;
                              % 1 --- реализация пакетом biblatex через движок
                              %       biber
}{}
\makeatother

%%% Вывод типов ссылок в библиографии %%%
\makeatletter
\@ifundefined{c@mediadisplay}{
  \newcounter{mediadisplay}
  \setcounter{mediadisplay}{2}   % 0 --- не делать ничего; надписи [Текст] и
                                 %       [Эл. ресурс] будут выводиться только в ссылках с
                                 %       заполненным полем `media`;
                                 % 1 --- автоматически добавлять надпись [Текст] к ссылкам с
                                 %       незаполненным полем `media`; таким образом, у всех
                                 %       источников будет указан тип, что соответствует
                                 %       требованиям ГОСТ
                                 % 2 --- автоматически удалять надписи [Текст], [Эл. Ресурс] и др.;
                                 %       не соответствует ГОСТ
                                 % 3 --- автоматически удалять надпись [Текст];
                                 %       не соответствует ГОСТ
                                 % 4 --- автоматически удалять надпись [Эл. Ресурс];
                                 %       не соответствует ГОСТ
}{}
\makeatother

%%% Проверка используемого TeX-движка %%%
\newif\ifxetexorluatex   % определяем новый условный оператор (http://tex.stackexchange.com/a/47579)
\ifxetex
    \xetexorluatextrue
\else
    \ifluatex
        \xetexorluatextrue
    \else
        \xetexorluatexfalse
    \fi
\fi

\let\CYRDZE\relax
\usepackage{etoolbox}
\providebool{presentation}
\usepackage{hyperref}
\usepackage{geometry}
\usepackage{caption}
\usepackage{subcaption}

%%% Математические пакеты %%%
\usepackage{amsthm,amsmath,amscd}   % Математические дополнения от AMS
\usepackage{amsfonts,amssymb}       % Математические дополнения от AMS
\usepackage{mathtools}              % Добавляет окружение multlined
\usepackage{xfrac}                  % Красивые дроби

%%% Счётчики %%%
\usepackage{aliascnt}
\usepackage[figure,table]{totalcount}   % Счётчик рисунков и таблиц
\usepackage{totcount}   % Пакет создания счётчиков на основе последнего номера
                        % подсчитываемого элемента (может требовать дважды
                        % компилировать документ)
\usepackage{totpages}   % Счётчик страниц, совместимый с hyperref (ссылается
                        % на номер последней страницы). Желательно ставить
                        % последним пакетом в преамбуле

%%% Продвинутое управление групповыми ссылками (пока только формулами) %%%
\ifpresentation
\else
    \usepackage[russian]{cleveref} % cleveref имеет сложности со считыванием
    % языка из babel. Такое решение русификации вывода выбрано вместо
    % определения в documentclass из опасности что-то лишнее передать во все
    % остальные пакеты, включая библиографию.

    % Добавление возможности использования пробелов в \labelcref
    % https://tex.stackexchange.com/a/340502/104425
    \usepackage{kvsetkeys}
    \makeatletter
    \let\org@@cref\@cref
    \renewcommand*{\@cref}[2]{%
        \edef\process@me{%
            \noexpand\org@@cref{#1}{\zap@space#2 \@empty}%
        }\process@me
    }
    \makeatother
\fi

\usepackage{placeins} % для \FloatBarrier

%%%% Установки для размера шрифта 14 pt %%%%
%% Формирование переменных и констант для сравнения (один раз для всех подключаемых файлов)%%
%% должно располагаться до вызова пакета fontspec или polyglossia, потому что они сбивают его работу
\newlength{\curtextsize}
\newlength{\bigtextsize}
\setlength{\bigtextsize}{13.9pt}

\makeatletter
%\show\f@size    % неплохо для отслеживания, но вызывает стопорение процесса,
                 % если документ компилируется без команды  -interaction=nonstopmode
\setlength{\curtextsize}{\f@size pt}
\makeatother


%%% Оформление абзацев %%%
\ifpresentation
\else
    \indentafterchapter     % Красная строка после заголовков типа chapter
    \usepackage{indentfirst}
\fi

%%% Цвета %%%
\ifpresentation
\else
    \usepackage[dvipsnames, table]{xcolor} % Совместимо с tikz
\fi

%%% Таблицы %%%
\usepackage{longtable} % Длинные таблицы
\usepackage{multirow,makecell}   % Улучшенное форматирование таблиц
\usepackage{tabulary,tabularray} % Таблицы с автоматически подбирающейся
                                 % шириной столбцов
\UseTblrLibrary{booktabs}
\ExplSyntaxOn% define \IfTokenListEmpty to use \captionof with tabularray
\prg_generate_conditional_variant:Nnn \tl_if_empty:n { e } { TF }
\let \IfTokenListEmpty = \tl_if_empty:eTF
\ExplSyntaxOff

\usepackage{threeparttable}      % автоматический подгон ширины подписи таблицы

%%% Общее форматирование
%\usepackage{soul}% Поддержка переносоустойчивых подчёркиваний и зачёркиваний
\usepackage{icomma}  % Запятая в десятичных дробях

%%% Для добавления Стр. над номерами страниц в оглавлении
%%% http://tex.stackexchange.com/a/306950
\usepackage{afterpage}

%%% Списки %%%
\usepackage{enumitem}

%%% Оформление списка обозначений
\usepackage[intoc]{nomencl}
% Переопределения названий для nomencl. Так как опция russian не для utf8
\renewcommand{\nomname}{Наиболее употребительные обозначения}%
\renewcommand{\eqdeclaration}[1]{, см.~(#1)}%
\renewcommand{\pagedeclaration}[1]{, стр.~#1}%
\renewcommand{\nomAname}{Латинские буквы}%
\renewcommand{\nomGname}{Греческие буквы}%
\renewcommand{\nomXname}{Верхние индексы}%
\renewcommand{\nomZname}{Индексы}%

\makenomenclature
\setlength{\nomitemsep}{-.8\parsep}

\input{Dissertation/userpackages}
\usepackage{polyglossia}
\setmainlanguage[babelshorthands=true]{russian}
\setotherlanguage{english}

%%% Режим черновика %%%
\makeatletter
\@ifundefined{c@draft}{
  \newcounter{draft}
  \setcounter{draft}{0}  % 0 --- чистовик (максимальное соблюдение ГОСТ)
                         % 1 --- черновик (отклонения от ГОСТ, но быстрая
                         %       сборка итоговых PDF)
}{}
\makeatother


%%% Библиография %%%
\makeatletter
\@ifundefined{c@bibliosel}{
  \newcounter{bibliosel}
  \setcounter{bibliosel}{1}   % 0 --- встроенная реализация с загрузкой файла
                              %       через движок bibtex8;
                              % 1 --- реализация пакетом biblatex через движок
                              %       biber
}{}
\makeatother

%%% Вывод типов ссылок в библиографии %%%
\makeatletter
\@ifundefined{c@mediadisplay}{
  \newcounter{mediadisplay}
  \setcounter{mediadisplay}{2}   % 0 --- не делать ничего; надписи [Текст] и
                                 %       [Эл. ресурс] будут выводиться только в ссылках с
                                 %       заполненным полем `media`;
                                 % 1 --- автоматически добавлять надпись [Текст] к ссылкам с
                                 %       незаполненным полем `media`; таким образом, у всех
                                 %       источников будет указан тип, что соответствует
                                 %       требованиям ГОСТ
                                 % 2 --- автоматически удалять надписи [Текст], [Эл. Ресурс] и др.;
                                 %       не соответствует ГОСТ
                                 % 3 --- автоматически удалять надпись [Текст];
                                 %       не соответствует ГОСТ
                                 % 4 --- автоматически удалять надпись [Эл. Ресурс];
                                 %       не соответствует ГОСТ
}{}
\makeatother


\input{common/newnames}         % Новые переменные, для всего проекта
%%% Основные сведения %%%
\newcommand{\thesisAuthorLastName}{Ходюня}
\newcommand{\thesisAuthorOtherNames}{Николай Дмитриевич}
\newcommand{\thesisAuthorInitials}{Н.\,Д.}
\newcommand{\thesisAuthor}             % Диссертация, ФИО автора
{%
    \texorpdfstring{% \texorpdfstring takes two arguments and uses the first for (La)TeX and the second for pdf
        \thesisAuthorLastName~\thesisAuthorOtherNames% так будет отображаться на титульном листе или в тексте, где будет использоваться переменная
    }{%
        \thesisAuthorLastName, \thesisAuthorOtherNames% эта запись для свойств pdf-файла. В таком виде, если pdf будет обработан программами для сбора библиографических сведений, будет правильно представлена фамилия.
    }
}
\newcommand{\thesisAuthorShort}        % Диссертация, ФИО автора инициалами
{\thesisAuthorInitials~\thesisAuthorLastName}
%\newcommand{\thesisUdk}                % Диссертация, УДК
%{\fixme{xxx.xxx}}
\newcommand{\thesisTitle}              % Диссертация, название
{Неассоциативные обертывающие алгебры нильтреугольных алгебр Шевалле}
\newcommand{\thesisSpecialtyNumber}    % Диссертация, специальность, номер
{01.01.06}
\newcommand{\thesisSpecialtyTitle}     % Диссертация, специальность, название (название взято с сайта ВАК для примера)
{Математическая логика, алгебра и теория чисел}
\newcommand{\thesisDegree}             % Диссертация, ученая степень
{кандидата физико-математических наук}
\newcommand{\thesisDegreeShort}        % Диссертация, ученая степень, краткая запись
{канд. физ.-мат. наук}
\newcommand{\thesisCity}               % Диссертация, город написания диссертации
{Красноярск}
\newcommand{\thesisYear}               % Диссертация, год написания диссертации
{\the\year}
\newcommand{\thesisOrganization}       % Диссертация, организация
{Федеральное государственное автономное образовательное учреждение высшего
образования <<Сибирский Федеральный Университет>>}
\newcommand{\thesisOrganizationShort}  % Диссертация, краткое название организации для доклада
{СФУ}

\newcommand{\thesisInOrganization}     % Диссертация, организация в предложном падеже: Работа выполнена в ...
{ФГАОУ ВО <<Сибирский Федеральный Университет>>}

\newcommand{\supervisorDead}{}           % Рисовать рамку вокруг фамилии
\newcommand{\supervisorFio}              % Научный руководитель, ФИО
{Левчук Владимир Михайлович}
\newcommand{\supervisorRegalia}          % Научный руководитель, регалии
{доктор физ.-мат.~наук, профессор}
\newcommand{\supervisorFioShort}         % Научный руководитель, ФИО
{В.\:Л.~Левчук}
\newcommand{\supervisorRegaliaShort}     % Научный руководитель, регалии
{д-р физ.-мат. наук,~профессор}

%\newcommand{\supervisorTwoDead}{}        % Рисовать рамку вокруг фамилии
\newcommand{\supervisorTwoFio}           % Второй научный руководитель, ФИО
{Сулейманова Галина Сафиуллановна}
\newcommand{\supervisorTwoRegalia}       % Второй научный руководитель, регалии
{доктор физ.-мат.~наук, профессор}
\newcommand{\supervisorTwoFioShort}      % Второй научный руководитель, ФИО
{Г.\:С.~Сулейманова}
\newcommand{\supervisorTwoRegaliaShort}  % Второй научный руководитель, регалии
{д-р физ.-мат. наук,~профессор}

\newcommand{\opponentOneFio}           % Оппонент 1, ФИО
{\fixme{Фамилия Имя Отчество}}
\newcommand{\opponentOneRegalia}       % Оппонент 1, регалии
{\fixme{доктор физико-математических наук, профессор}}
\newcommand{\opponentOneJobPlace}      % Оппонент 1, место работы
{\fixme{Не очень длинное название для места работы}}
\newcommand{\opponentOneJobPost}       % Оппонент 1, должность
{\fixme{старший научный сотрудник}}

\newcommand{\opponentTwoFio}           % Оппонент 2, ФИО
{\fixme{Фамилия Имя Отчество}}
\newcommand{\opponentTwoRegalia}       % Оппонент 2, регалии
{\fixme{кандидат физико-математических наук}}
\newcommand{\opponentTwoJobPlace}      % Оппонент 2, место работы
{\fixme{Основное место работы c длинным длинным длинным длинным названием}}
\newcommand{\opponentTwoJobPost}       % Оппонент 2, должность
{\fixme{старший научный сотрудник}}

%% \newcommand{\opponentThreeFio}         % Оппонент 3, ФИО
%% {\fixme{Фамилия Имя Отчество}}
%% \newcommand{\opponentThreeRegalia}     % Оппонент 3, регалии
%% {\fixme{кандидат физико-математических наук}}
%% \newcommand{\opponentThreeJobPlace}    % Оппонент 3, место работы
%% {\fixme{Основное место работы c длинным длинным длинным длинным названием}}
%% \newcommand{\opponentThreeJobPost}     % Оппонент 3, должность
%% {\fixme{старший научный сотрудник}}

\newcommand{\leadingOrganizationTitle} % Ведущая организация, дополнительные строки. Удалить, чтобы не отображать в автореферате
{\fixme{Федеральное государственное бюджетное образовательное учреждение высшего
профессионального образования с~длинным длинным длинным длинным названием}}

\newcommand{\defenseDate}              % Защита, дата
{\fixme{DD mmmmmmmm YYYY~г.~в~XX часов}}
\newcommand{\defenseCouncilNumber}     % Защита, номер диссертационного совета
{\fixme{Д\,123.456.78}}
\newcommand{\defenseCouncilTitle}      % Защита, учреждение диссертационного совета
{\fixme{Название учреждения}}
\newcommand{\defenseCouncilAddress}    % Защита, адрес учреждение диссертационного совета
{\fixme{Адрес}}
\newcommand{\defenseCouncilPhone}      % Телефон для справок
{\fixme{+7~(0000)~00-00-00}}

\newcommand{\defenseSecretaryFio}      % Секретарь диссертационного совета, ФИО
{\fixme{Фамилия Имя Отчество}}
\newcommand{\defenseSecretaryRegalia}  % Секретарь диссертационного совета, регалии
{\fixme{д-р~физ.-мат. наук}}            % Для сокращений есть ГОСТы, например: ГОСТ Р 7.0.12-2011 + http://base.garant.ru/179724/#block_30000

\newcommand{\synopsisLibrary}          % Автореферат, название библиотеки
{\fixme{Название библиотеки}}
\newcommand{\synopsisDate}             % Автореферат, дата рассылки
{\fixme{DD mmmmmmmm}\the\year~года}

% To avoid conflict with beamer class use \providecommand
\providecommand{\keywords}%            % Ключевые слова для метаданных PDF диссертации и автореферата
{}
             % Основные сведения
\setmonofont{Courier New}                          % моноширинный шрифт
\newfontfamily\cyrillicfonttt{Courier New}         % моноширинный шрифт для кириллицы
\defaultfontfeatures{Ligatures=TeX}                % стандартные лигатуры TeX, замены нескольких дефисов на тире и т. п. Настройки моноширинного шрифта должны идти до этой строки, чтобы при врезках кода программ в коде не применялись лигатуры и замены дефисов
\setmainfont{Times New Roman}                      % Шрифт с засечками
\newfontfamily\cyrillicfont{Times New Roman}       % Шрифт с засечками для кириллицы
\setsansfont{Arial}                                % Шрифт без засечек
\newfontfamily\cyrillicfontsf{Arial}               % Шрифт без засечек для кириллицы

\input{common/styles}           % Стили общие для диссертации и автореферата
%-------------------------------------------------
\numberwithin{equation}{section}
%-------------------------------------------------
\theoremstyle{plain}
\newtheorem{theorem}{Теорема}
\newtheorem{corollary}{Следствие}[theorem]
\newtheorem{lemma}{Лемма}[section]
\newtheorem{proposition}{Предложение}
%-------------------------------------------------
\theoremstyle{definition}
\newtheorem{definition}{Определение}
%\newtheorem{proof}{Доказательство}\def\theproof{}
\newtheorem{remark}{Замечание}

\DeclareMathOperator*{\res}{res}
\DeclareMathOperator*{\Ann}{Ann}

\input{Dissertation/disstyles}
\input{Dissertation/userstyles}
\input{biblio/biblatex}

\begin{document}
%%% Переопределение именований типовых разделов
% https://tex.stackexchange.com/a/156050
\gappto\captionsrussian{\input{common/renames}} % for polyglossia and babel
\input{common/renames}

\include{Dissertation/title}           % Титульный лист
\include{Dissertation/contents}        % Оглавление
\ifnumequal{\value{contnumfig}}{1}{}{\counterwithout{figure}{chapter}}
\ifnumequal{\value{contnumtab}}{1}{}{\counterwithout{table}{chapter}}
\chapter*{Введение}                         % Заголовок
\addcontentsline{toc}{chapter}{Введение}    % Добавляем его в оглавление

\newcommand{\actuality}{}
\newcommand{\progress}{}
\newcommand{\aim}{{\textbf\aimTXT}}
\newcommand{\tasks}{\textbf{\tasksTXT}}
\newcommand{\novelty}{\textbf{\noveltyTXT}}
\newcommand{\influence}{\textbf{\influenceTXT}}
\newcommand{\methods}{\textbf{\methodsTXT}}
\newcommand{\defpositions}{\textbf{\defpositionsTXT}}
\newcommand{\reliability}{\textbf{\reliabilityTXT}}
\newcommand{\probation}{\textbf{\probationTXT}}
\newcommand{\contribution}{\textbf{\contributionTXT}}
\newcommand{\publications}{\textbf{\publicationsTXT}}

Ассоциативное кольцо $A$ всегда превращается в кольцо Ли $A^{(-)}$, если
умножение в $A$ заменим новым $[a,b]\coloneq ab - ba$ (коммутирование).
А.\:А.~Альберт \cite{albert48} называет произвольную алгебру $A$ (не
обязательно ассоциативную) \emph{Ли-допустимой}, когда $A^{(-)}$ есть алгебра
Ли; см. также \cite{myung72}. Согласно \cite{DAN2018} и ~\cite{imm2020}, $A$
называется \emph{точной обертывающей} алгебры Ли $L$, если $A^{(-)}\simeq L$.

\medskip

Алгебру Шевалле над полем $K$ характеризуют системой корней $\Phi$ и базой,
состоящей из элементов $e_r$ $(r \in \Phi)$ и подходящей базы подалгебры
Картана. Как показано в \cite[\S\ 4.2]{carter72}, структурные константы базы
Шевалле однозначно определяет их выбор для \emph{нильтреугольной} подалгебры
$N\Phi(K)$ с базой $\{e_r\ | \ r \in \Phi^+ \}$. Унипотентная подгруппа
$U\Phi(K)$ группы Шевалле типа $\Phi$ над $K$ представлена в \cite{vl90a_ru}
присоединенной группой на $N\Phi(K)$. Как правило, ее нормальные подгруппы --
это, в точности, идеалы кольца Ли $N\Phi(K)$, \cite{LevSul-DAN, JA-12}.

Для корней считаем $s\geq r$, когда в разложении $s-r$ по базе $\Pi$ в $\Phi^+$
все коэффициенты неотрицательны. Корни $r$ и $s$ в $\Phi^+$ назовем
\emph{инцидентными}, если $s \geq r$ или $r\geq s$. Любое множество
$\mathcal{L}$ попарно неинцидентных корней в $\Phi^+$ называем \emph{множеством
    углов в $\Phi^+$}. Выделим в $N\Phi(K)$ идеалы
%
\[T(r)=\sum_{s \geq r}Ke_s,\quad Q(r)=\sum_{s>r}Ke_s\quad (r\in
    \Phi^+),\quad Q(\mathcal{L})= \sum_{r \in \mathcal{L}}Q(r).\]
%
Если $H \subseteq T(\mathcal{L})\coloneq\sum_{r \in \mathcal{L}} T(r)$ и включение
нарушается при любой замене $T(r)$ на $Q(r)$ в сумме, то множество
$\mathcal{L}= \mathcal{L}(H)$ определено однозначно и, согласно \cite[п.
    3]{DAN2018}, называется \emph{множеством углов в $H$}.

Идеал $H$ кольца Ли $N\Phi(K)$ называем \emph{стандартным}, если $Q
    (\mathcal{L}(H))\subset H$.

\medskip

Известные взаимосвязанные перечисления идеалов колец Ли $N\Phi(K)$ и нормальных
подгрупп групп $U\Phi(K)$ редуцировались к перечислениям идеалов вида
$T(\mathcal{L})$ (\cite[следствие 4.3]{vL74}, \cite[теорема
    2.1.2]{Egorychev84}, \cite{Eg-Lev-1996, eS05} и др.) и, таким образом, к
перечислениям путей в различных решетках.

Для классических типов в 2001 году в~\cite{sigsam2001} записана, как
проблема~1, задача

\medskip

\noindent\textbf{(A)} \emph{Найти число стандартных идеалов алгебры Ли
    $N\Phi(K)$ над конечным полем $K$.}

\medskip

В \S~\ref{sec:Problem-1} проблему~1 из \cite{sigsam2001} решает теорема
\ref{th:Enum-St-Id}, анонсированная ранее \cite[Теорема 3]{DAN2018}. Для
исключительных типов задачу~\textbf{(A)} решает теорема \ref{th:exhls}.

\medskip

При фиксированном $\Phi$ для алгебры Ли $N\Phi(K)$ точные обертывающие алгебры
$R_{\Phi}$ и их число выявляет предложение \ref{p:mult-enveloping} в
\S~\ref{sec:enveloping}. Известно, что одну из них для $\Phi$ типа $A_{n-1}$
представляет алгебра $NT(n, K)$ нильтреугольных $n\times n$ матриц над $K$; все
идеалы алгебры и кольца $NT(n, K)$ стандартны \cite{Dub-Perl-1951, vL76}.

Алгебру $R_{\Phi}$ называем стандартной, если все ее идеалы стандартны. В силу
теоремы \ref{th:classical-Exc-RDn} и предложения \ref{p:Exc-Dn-En} (см. также
\cite{DAN2018} и \cite{jsfu2018}), стандартная алгебра $R_{\Phi}$ существует
для всех типов, исключая тип $D_n$ $(n \geq 4)$ и $E_n\ (n=6,7,8)$.

Алгебра $NT(n,K)$ нижних нильтреугольных (т.\:е. с нулями на главной диагонали
и над ней) $n \times n$ матриц над полем $K$, с точностью до изоморфизма,
оказывается единственной ассоциативной обертывающей алгеброй $R_{\Phi}$ типа
$A_{n-1}$ (теорема \ref{th:Stand-An} в \S~\ref{sec:uniqtheorems} и
\cite{imm2020}); более слабым является условие стандартности.

Вопрос об условиях однозначности неассоциативной обертывающей алгебры отмечал
И.\:П.~Шестаков на конференции в 2017 году (см. \cite{modernWorldConf2017}).
Для классических типов этот вопрос исследуется в \S~\ref{sec:uniqtheorems}
(теоремы \ref{th:Stand-An}, \ref{th:Stand-Bn} и замечание
\ref{remark:Stand-Cn}).

\medskip

Лемма~\ref{l:LieIdeal} и теорема~\ref{th:ExIdForm} характеризуют нестандартные
идеалы единственной нестандартной алгебры $RD_n(K)$ специальным набором
параметров.

\medskip

В работе используем стандартные обозначения \cite{carter72}, \cite{nB72}:
$\Phi^+$ "--- система поло\-жительных корней, $\Pi$ "--- ее база, $\rho$ "---
максимальный в $\Phi^+$ корень, $ht(r)$ "--- высота корня $r$. Число Кокстера
$h=h(\Phi)$ системы $\Phi$ равно $ht(\rho)+1$. \nomenclature{$\Phi^+$}{система
    поло\-жительных корней} \nomenclature{$\Pi$}{база системы корней}
\nomenclature{$\rho$}{максимальный в $\Phi^+$ корень}
\nomenclature{$ht(r)$}{высота корня $r$} \nomenclature{h(\Phi)}{Число Кокстера
    системы $\Phi$ (равно $ht(\rho)+1$)}

{\aim} данной работы является исследование записанных в 2001 году проблем
комбинаторного перечисления идеалов алгебры Ли $N\Phi(K)$
\cite[Проблемы 1 и 2]{sigsam2001} и нахождение условия однозначности
ее точной обертывающей алгебры.

{\novelty}
\begin{enumerate}[beginpenalty=10000] % https://tex.stackexchange.com/a/476052/104425
    \item Впервые \ldots
    \item Впервые \ldots
    \item Было выполнено оригинальное исследование \ldots
\end{enumerate}

{\influence} Все основные результаты диссертации являются новыми. Работа носит теоретический характер.

{\methods} Развиваемый Г.\:П.~Егорычевым с 1970-х годов метод интегрального представления
и вычисления комбинаторных сумм (метод коэффициентов) находит приложения в
многочисленных задачах алгебры, комбинаторного анализа и других областей
математики, \cite{Egorychev84, Eg-2009, Leontiev2001, Riedel-2023,
    Eg-2013}. Комбинаторная теорема \ref{th:Enum-Dn}, завершающая перечисление всех
идеалов обертывающих алгебр классических типов, впервые применяется для
вычисления 3-кратной комбинаторной суммы с $q$-биномиальными коэффициентами.
См. также замечание~\ref{remark:problem-2}.

    {\probation} Результаты диссертационной работы докладывались на заседаниях Красноярского
алгебраического семинара (2016--2020\,гг.), на cеминаре им.~Н.\:А. Вавилова
(СПбГУ, г.~Санкт-Петербург, 8 декабря 2017\,г.) и апробировались на следующих
конференциях:
\begin{enumerate}
    \item Международная научная конференция студентов, аспирантов и молодых учёных
          <<Проспект Свободный>>, 2016\,г., г.~Красноярск.
    \item Международная конференция, посвященная 70-летию В.\,М.~Левчука <<Алгебра и
          Логика: Теория и Приложения>>, 2016\,г., г.~Красноярск.
    \item Международная конференция, посвященная 60-летию Института математики
          им.~С.\:Л.~Соболева <<Математика в современном мире>>, 2017\,г.,
          г.~Новосибирск.
    \item Международная алгебраическая конференция, посвященная 110-летию со дня рождения
          профессора А.\:Г.~Куроша, 2018\,г., г.~Москва.
    \item Международная конференция <<Мальцевские чтения>>, 2016 и 2023\,г.,
          г.~Новосибирск.
\end{enumerate}

\begin{refsection}[bl-author]
    % Это refsection=1.
    % Процитированные здесь работы:
    %  * подсчитываются, для автоматического составления фразы "Основные результаты ..."
    %  * попадают в авторскую библиографию, при usefootcite==0 и стиле `\insertbiblioauthor` или `\insertbiblioauthorgrouped`
    %  * нумеруются там в зависимости от порядка команд `\printbibliography` в этом разделе.
    %  * при использовании `\insertbiblioauthorgrouped`, порядок команд `\printbibliography` в нём должен быть тем же (см. biblio/biblatex.tex)
    %
    % Невидимый библиографический список для подсчёта количества публикаций:
    \phantom{\printbibliography[heading=nobibheading, section=1, env=countauthorvak,          keyword=biblioauthorvak]%
        \printbibliography[heading=nobibheading, section=1, env=countauthorwos,          keyword=biblioauthorwos]%
        \printbibliography[heading=nobibheading, section=1, env=countauthorscopus,       keyword=biblioauthorscopus]%
        \printbibliography[heading=nobibheading, section=1, env=countauthorconf,         keyword=biblioauthorconf]%
        \printbibliography[heading=nobibheading, section=1, env=countauthorother,        keyword=biblioauthorother]%
        \printbibliography[heading=nobibheading, section=1, env=countregistered,         keyword=biblioregistered]%
        \printbibliography[heading=nobibheading, section=1, env=countauthorpatent,       keyword=biblioauthorpatent]%
        \printbibliography[heading=nobibheading, section=1, env=countauthorprogram,      keyword=biblioauthorprogram]%
        \printbibliography[heading=nobibheading, section=1, env=countauthor,             keyword=biblioauthor]%
        \printbibliography[heading=nobibheading, section=1, env=countauthorvakscopuswos, filter=vakscopuswos]%
        \printbibliography[heading=nobibheading, section=1, env=countauthorscopuswos,    filter=scopuswos]}%
    %
    \nocite{*}%
    %
    {\publications} Основные результаты по теме диссертации изложены в~\arabic{citeauthor}~печатных изданиях,
    \arabic{citeauthorvak} из которых изданы в журналах, рекомендованных ВАК%
    \ifnum \value{citeauthorscopuswos}>0%
        , \arabic{citeauthorscopuswos} "--- в~периодических научных журналах, индексируемых Web of~Science и Scopus%
    \fi%
    \ifnum \value{citeauthorconf}>0%
        , \arabic{citeauthorconf} "--- в~тезисах докладов.
    \else%
        .
    \fi%
    % К публикациям, в которых излагаются основные научные результаты диссертации на соискание учёной
    % степени, в рецензируемых изданиях приравниваются патенты на изобретения, патенты (свидетельства) на
    % полезную модель, патенты на промышленный образец, патенты на селекционные достижения, свидетельства
    % на программу для электронных вычислительных машин, базу данных, топологию интегральных микросхем,
    % зарегистрированные в установленном порядке.(в ред. Постановления Правительства РФ от 21.04.2016 N 335)
\end{refsection}%
\begin{refsection}[bl-author]
    % Это refsection=2.
    % Процитированные здесь работы:
    %  * попадают в авторскую библиографию, при usefootcite==0 и стиле `\insertbiblioauthorimportant`.
    %  * ни на что не влияют в противном случае
    \nocite{vmj2015}%vak
    \nocite{jsfu2018}%patent
    \nocite{imm2020}%program
    \nocite{smj2023}%other
\end{refsection}%
%
% Всё, что вне этих двух refsection, это refsection=0,
%  * для диссертации - это нормальные ссылки, попадающие в обычную библиографию
%  * для автореферата:
%     * при usefootcite==0, ссылка корректно сработает только для источника из `external.bib`. Для своих работ --- напечатает "[0]" (и даже Warning не вылезет).
%     * при usefootcite==1, ссылка сработает нормально. В авторской библиографии будут только процитированные в refsection=0 работы.
 % Характеристика работы по структуре во введении и в автореферате не отличается (ГОСТ Р 7.0.11, пункты 5.3.1 и 9.2.1), потому её загружаем из одного и того же внешнего файла, предварительно задав форму выделения некоторым параметрам

\textbf{Объем и структура работы.} Диссертация состоит из~введения,
\formbytotal{totalchapter}{глав}{ы}{}{},
заключения и
\formbytotal{totalappendix}{приложен}{ия}{ий}{}.
%% на случай ошибок оставляю исходный кусок на месте, закомментированным
%Полный объём диссертации составляет  \ref*{TotPages}~страницу
%с~\totalfigures{}~рисунками и~\totaltables{}~таблицами. Список литературы
%содержит \total{citenum}~наименований.
%
Полный объём диссертации составляет \formbytotal{TotPages}{страниц}{у}{ы}{},
включая \formbytotal{totalcount@figure}{рисун}{ок}{ка}{ков} и
\formbytotal{totalcount@table}{таблиц}{у}{ы}{}. Список литературы содержит
\formbytotal{citenum}{наименован}{ие}{ия}{ий}.
%%

Автор благодарен научным руководителям профессору Левчуку Владимиру Михайловичу
и профессору Сулеймановой Галине Сафиуллановне за постановку задач и внимание к
работе. Признателен сотрудникам кафедры алгебры и математической логики и
Института математики и фундаментальной информатики СФУ за хорошие условия
работы над диссертацией.
    % Введение
\ifnumequal{\value{contnumfig}}{1}{\counterwithout{figure}{chapter}
}{\counterwithin{figure}{chapter}}
\ifnumequal{\value{contnumtab}}{1}{\counterwithout{table}{chapter}
}{\counterwithin{table}{chapter}}

\chapter{Неассоциативные обертывающие алгебры нильтреугольных алгебр Шевалле}\label{ch:enveloping}

\section{Точные обертывающие алгебры и стандартные идеалы}\label{sec:enveloping}

Отметим, что в теории алгебр Ли с самого ее основания эффективно используются
универсальные ассоциативные обертывающие алгебры, см. теорему Пуанкаре"--~
Биркгофа"--~Витта \cite[глава 1, \S\ 1]{Kaplansky1974} и
\cite[\S~V.9]{Kurosh1973}.

По определению \cite{DAN2018}, если алгебра $A$ есть точная обертывающая
алгебры Ли $L$, то алгебры $L$ и $A^{(-)}$ изоморфны. Поэтому алгебру Ли $L$ и
ее точную обертывающую алгебру $A$ можно определять структурными константами в
одной базе, в отличие от универсальной ассоциативной обертывающей алгебры.

Точные обертывающие алгебры определенных подалгебр алгебр Шевалле указаны в
\cite{DAN2018}. Нам потребуются предварительные сведения.

\medskip

В теории Картана"--~Киллинга всякую простую комплексную конечномерную алгебру
Ли $L$ ассоциируют с единственной (с точностью до эквивалентности) неразложимой
системой корней $\Phi$ евклидова пространства $V$, построенного на подалгебре
Картана \cite[Глава 3]{carter72}. Выбор простых корней или базы $\Pi$ в $\Phi$
определяет линейное упорядочение $\prec$ на $V$ с множеством $V^+$ векторов
$v\succ 0$ и систему положительных корней $\Phi^+=V^+ \cap \Phi\supseteq \Pi$.

Элементы $e_r$ $(r \in \Phi)$ и подходящая база подалгебры Картана алгебры Ли
$L=\mathcal{L} (\Phi, \mathbb{C})$ дают \cite{Chev-55} базу Шевалле с
целочисленными структурными константами, приводящую к алгебре Ли $\mathcal{L}
    (\Phi, K)$ над любым полем $K$ (алгебра Шевалле). Ее подалгебру $N\Phi(K)$ с
базой $\{e_r\ | \ r \in \Phi^+ \}$ называем \emph{нильтреугольной}.

По теореме Шевалле о базе \cite[Теорема 4.2.1]{carter72} при $r,s\in \Phi$
имеем
%
\[  [e_r ,e_s] = N_{rs}e_{r+s}=-[e_s ,e_r] \ \ (r+s\in \Phi),
    \qquad [e_r,e_s]=0 \  \  (r+s\notin \Phi \cup \{0\}),
\]
%
где либо $N_{rs}=\pm 1$, либо $|r|=|s|<|r+s|$ и $N_{rs}=\pm 2$, либо
$N_{rs}=\pm 2$ или $\pm 3$ и $\Phi$ типа $G_2$. Согласно \cite[Предложение
    4.2.2]{carter72}, знаки структурных констант $N_{rs}$ имеют определенный
произвол для алгебры Шевалле, а также для $N\Phi(K)$.

\medskip

Фиксируя выбор знаков структурных констант $N_{rs}$, через $R_{\Phi}$ обозначим
$K$-алгебру с базой $\{e_r\ | \ r \in \Phi^+\}$ и умножением:\ $e_re_s=0$ при
$r+s \not \in \Phi$ и

\[
    e_r e_s= e_{r+s},\qquad e_se_r=(1-N_{rs}) e_{r+s}\qquad (r,s,r+s\in \Phi^+,\ N_{rs}\geq 1).
\]

При фиксированном $\Phi$ точные обертывающие алгебры $R_{\Phi}$ и их число
выявляет

\begin{proposition}\label{p:mult-enveloping}
    %
    Для алгебры Ли $N\Phi(K)$ каждая алгебра $R_{\Phi}$ является точной
    обертывающей алгеброй и их число равно $2^{|\Phi^+ \setminus \Pi |}$.
\end{proposition}

\begin{proof}
    %
    Таблицы умножения базисных элементов $N\Phi(K)$ и любой алгебры
    $R_{\Phi}^{(-)}$, очевидно, совпадают. Поэтому, как и в {\cite[Предложение
        1]{DAN2018}}, алгебра $R_{\Phi}$ есть точная обертывающая алгебры Ли
    $N\Phi(K)$.

    Когда $r,s,r+s \in \Phi^+$ и $0\prec r \prec s$, пару корней $r, s$ называют
    \emph{специальной}, а также \emph{экстраспециальной}, если $r \preccurlyeq r_1$
    для каждой специальной пары $r_1, s_1$ с суммой $r_1+s_1 =r+s$. Согласно
    ~\cite[Предложение 4.2.2]{carter72}, знаки констант $N_{rs}$ можно выбрать для
    экстраспециальных пар $(r,s)$ произвольно, с точностью до изоморфизмов алгебры
    Ли $\mathcal{L} (\Phi, K)$ (аналогично, $N\Phi(K)$); тогда знаки остальных
    констант $N_{rs}$ определяются однозначно.

    Любую из построенных точных обертывающих алгебр $R_{\Phi}$ алгебры Ли
    $N\Phi(K)$ однозначно определяет выбор знака $\pm$ (одна из двух возможностей)
    константы $N_{rs}$ для каждой экстраспециальной пары корней $(r,s)$ в $\Phi^+$.

    С другой стороны, любой положительный непростой корень в $\Phi$ есть сумма
    $r+s$ единственной экстраспециальной пары корней $r,s$. Поэтому множество
    экстраспециальных пар взаимнооднозначно множеству $\Phi^+ \setminus \Pi$.

    Отсюда вытекает запись требуемого в предложении числа в виде $2^{|\Phi^+
                \setminus \Pi |}$.
\end{proof}

\begin{lemma}\label{l:Const-Non-assoc}
    %
    Все ненулевые структурные константы алгебр $R_{\Phi}$ в выбранной базе равны \
    $\pm 1$ для типа $\neq G_2$ и равны $1$, когда все корни в $\Phi$ одной длины.
    Если $\Phi$ имеет корни разных длин, то все алгебры $R_{\Phi}$ неассоциативные.
\end{lemma}

\begin{proof}
    %
    Когда система корней $\Phi$ имеет корни разных длин, найдутся корни $r, s \in
        \Phi^+$, для которых пересечение $\Phi\cap (Zr+ Zs)=\Psi$ есть подсистема
    корней типа $B_2$ или $G_2$. Определение алгебры $R_{\Phi}$ позволяет считать,
    что $\{r, s \}$ есть база в $\Psi$ и $2r+s$ "--- корень. Если $N_{rs} \geq 1$,
    то получаем
    \[
        e_r(e_re_s) =e_re_{r+s} = \pm e_{2r+s},\quad (e_re_r)e_s=0.
    \]
    Аналогично при $N_{sr}\ (=- N_{rs})\ \geq 1$ произведения $(e_se_r)e_r =e_{r+s}
        e_r = \pm e_{2r+s}$ и $e_s(e_re_r)=0$ не совпадают. Следовательно, алгебра
    $R_{\Phi}$ "--- неассоциативная.

    \medskip

    Первое утверждение в лемме сейчас сразу следует из определения алгебр
    $R_{\Phi}$. Таким образом, доказательство завершено.
\end{proof}

\medskip

Для каждого типа $\Phi$ мы исследуем (\S~\ref{sec:uniqtheorems}) условия
однозначности алгебр $R_{\Phi}$ и, взаимосвязано, структурных констант $N_{rs}$
алгебр Ли $N\Phi(K)$ над полем $K$.

\medskip

Идеал произвольного кольца $A$ всегда есть идеал и кольца $A^{(-)}$, так как
основные операции в $A^{(-)}$ производны от операций в $A$. Верно и обратное,
когда умножение в $A$ есть также производная операция от операций в $A^{(-)}$,
например, если для алгебры Ли $L$ с умножением $*$ выберем обертывающую алгебру
$A$ с умножением $\alpha \cdot \beta=(\alpha* \beta)t$, где $t$ "--- обратимый
скаляр.

\medskip

Как и во введении, используем отношения частичного порядка $r>s$ корней и
инцидентности ($r\geq s$ или $s\geq r$), понятие множества углов $\mathcal{L}=
    \mathcal{L}(H)$ подмножеств $H$ в $N\Phi(K)$, а также идеалы
%
$T(r), \ Q(r), \ T(\mathcal{L})$ и $Q(\mathcal{L}).$
%

\begin{definition}\label{def:stand-enval}
    Идеал $H$ кольца Ли $N\Phi(K)$ называем \emph{стандартным}, если $Q
        (\mathcal{L}(H))\subset H$. Точную обертывающую алгебру $R_{\Phi}$
    алгебры Ли $N\Phi(K)$  над полем $K$ называем \emph{стандартной},
    если все ее идеалы стандартны.
\end{definition}

Из свойств систем корней и определений легко вытекает

\begin{lemma}\label{l:stand-one-corner}
    В кольце $R_{\Phi}$ каждый идеал с единственным углом стандартен.
    Когда $\Phi$ ранга $2$, любая алгебра $R_{\Phi}$ стандартна.
\end{lemma}

\begin{proof}
    Доказательство первого утверждения леммы для идеала $H$ в $R_{\Phi}$ c
    множеством углов $\mathcal{L}(H)=\{r\}$ проводим с помощью индукции по
    $h(\Phi)-ht(r)$. Ясно, что $Q(\rho) = 0 \subset H \subseteq T(r)$. Пусть $r$ "---
    не максимальный корень. Тогда существует простой корень $p$ с условием $r+p \in
        \Phi^+$ и одно из произведений $e_pe_r$, $e_re_p$ равно $e_{r+p}$. Поэтому
    множество $(Ke_p)H + H(Ke_p)$ по модулю $Q(r+p)$ совпадает с $Ke_{r+p}$ и, по
    индукции, порождает идеал, содержащий $Q(r+p)$, откуда $T(r+p) \subseteq H$.
    Учитывая произвол в выборе $p$, получаем включение $Q(r) \subset H$.

    Второе утверждение леммы сразу следует из первого для случая систем корней
    $\Phi$ типа $A_2$, $B_2$ и $G_2$.
\end{proof}

Хорошо известно, что алгебра Ли $N\Phi(K)$ типа $A_{n-1}$ изоморфна алгебре Ли,
ассоциированной с алгеброй $NT(n, K)$ (нижних) нильтреугольных $n\times n$
матриц над $K$. Таким образом, для алгебры Ли $N\Phi(K)$ типа $A_{n-1}$
существует ассоциативная точная обёртывающая алгебра "--- алгебра $R=NT(n,K)$.

Здесь обычные матричные единицы $e_{ij} \ (1 \leq j < i \leq n)$ составляют
базу алгебры $R$, а также базу Шевалле из элементов $e_r= e_{ij}\ (r \in
    \Phi^+)$ алгебры Ли $R^{(-)}= N\Phi(K)$ после соответствующей нумерации корней
$r=r_{ij}$. В этом случае известна стандартность всех идеалов алгебры $R$ над
полем $K$ (Р.~Дюбиш и С.~Перлис \cite[Теорема 8]{Dub-Perl-1951} и даже идеалов
кольца $R$, \cite[Теорема 2]{vL76}).

Алгебры Ли $N\Phi(K)$ остальных классических типов заданы в \cite[Лемма
    2]{vl90a_ru} аналогично в базе Шевалле из "матричных единиц" \ $e_{iv}$ с
ограничениями
\[
    B_n: \ -i < v < i \leq n;\quad
    D_n: \ 1 \leq |v| < i \leq n;\quad
    C_n: \ -i \leq v < i \leq n, \ v \neq 0.
\]

К суммам двух корней, являющихся корнями, помимо сумм $r_{ij}+ r_{jv}=r_{iv}$
(аналогично типу $A_n$), здесь относятся еще суммы
\[
    r_{kv}+ r_{m,-v}=r_{k, -m}\quad (k>m>|v|),
\]
а для типа $C_n$ при $k =m>|v|\geq 1$ также суммы $r_{kv}+ r_{k,-v} =
    r_{k,-k}$.

Любой элемент из $N\Phi(K)$ представляется суммой $\sum a_{iv}e_{iv}$ и
$\Phi^+$-матрицей $||a_{iv}||$ над $K$ соответствующего типа. Так,
$B_n^+$-матрица имеет вид
%
\[a_{10}\]
\[a_{2,-1}\ a_{20}\ a_{21}\]
\[\dots\ \ \dots\ \ \dots\]
\[a_{n,-n+1}\ \dots\  a_{n,-1}\ a_{n0}\ a_{n1}\ \dots\ a_{n,n-1}.\]
%
Отбрасывая столбец с нулевым номером, получаем $D_n^+$-матрицу.

В \cite[лемма 2]{vl90a_ru} доказана

\begin{lemma}\label{l:StructConst-vl90a_ru}
    Знаки структурных констант $N_{rs}$ алгебры Ли  $N\Phi(K)$ можно
    выбрать так, что $[e_{ij},e_{jv}]=e_{iv}=-[e_{jv},e_{ij}]$ и
    выполняются равенства:
    \[
        [e_{jv},e_{i,-v}]=e_{i,-j}\quad (\Phi=B_n,D_n),\quad i>j>|v|>0;\]
    %
    \[[e_{i0},e_{j0}]=2e_{i,-j}\quad(\Phi=B_n),\quad
        [e_{ij},e_{i,-j}]=2e_{i,-i}\quad(\Phi=C_n),\quad i>j\geq1;\]
    %
    \[[e_{jm},e_{i,-m}]=[e_{im},e_{j,-m}]=e_{i,-j}\quad(\Phi=C_n),\quad
        i>j>m\geq1.\]
\end{lemma}

\medskip

Соответственно типу ${\Phi}$ для алгебр Ли $N\Phi(K)$ используем обозначения
$NB_n(K)$, $NC_n(K)$ и так далее. Очевидным следствием
леммы~\ref{l:StructConst-vl90a_ru} является

\begin{lemma}\label{l:ReductBnCnDn}
    Фактор-алгебра $ND_n(K)/T(r)$ изоморфна $NT(n,K)^{(-)}$ точно для
    двух корней $r= r_{2,-1}$ и $r=r_{21}$ при $n>4$, и точно для трех
    простых корней $r$ при $n=4$.
    Кроме того, $NB_n(K)/T(r) \simeq NT(n+ 1,K)^{(-)}$ и $NC_n(K)/T(r) \simeq NT(n+
        1,K)^{(-)}$ для единственного корня $r=r_{2,-1}$ и $r=r_{2,-2}$ соответственно.
\end{lemma}

\begin{remark}\label{remark:A3-D3}
    Для ${\Phi}$ типа $B_n$ и $C_n$ произвольная обертывающая алгебра
    $R_{\Phi}$ алгебры Ли $N\Phi(K)$ имеет $K$-подмодуль $M_\Phi$ с
    базой $\{e_{iv}\ |\ 1 \leq |v| < i \leq n\}$. Он не является
    подалгеброй для ${\Phi}$ типа $C_n\ (n>1)$, а в алгебре $R_{\Phi}$
    типа $B_n$ есть подалгебра, изоморфная обертывающей алгебре алгебры Ли $ND_n(K)$.
\end{remark}

Точные обертывающие алгебры $R_{\Phi}$ классического типа с выбором знаков
структурных констант по лемме \ref{l:StructConst-vl90a_ru} обозначаем
соответственно типу
%
\begin{equation}\label{eq:RAn-RDn}
    RA_n(K)=NT(n+1, K),\ \ RB_n(K),\ \ RC_n(K),\ \ RD_n(K).
\end{equation}
%
\begin{lemma}\label{l:mult-RAn-RDn}
    %
    В алгебрах $R_{\Phi}$ из списка \eqref{eq:RAn-RDn} умножение определяют правила
    \[
        e_{ij}e_{jv}= e_{iv},\quad e_{iu}e_{jv}=0\quad (u\neq j, \ u\neq -v);
    \]
    %
    \[
        \Phi = B_n, D_n:\ \
        e_{jv}e_{i,-v}=e_{i,-j}\ \ (i>j>\mid v \mid > 0),\quad
        e_{iv}e_{j,-v}=0\ \ (i\geq j>\mid v \mid > 0);
    \]
    %
    \[
        \Phi=C_n:\ \ e_{ij}e_{i,-j}=-e_{i,-j}e_{ij}=e_{i,-i}\quad (i>j\geq 1),
    \]
    %
    \[
        e_{jm}e_{i,-m} = e_{im}e_{j,-m} = e_{i,-j},\quad
        e_{i,-m}e_{jm}=e_{j,-m}e_{im}=0\quad (i>j>m \geq 1);
    \]
    %
    \[
        \Phi = B_n:\qquad e_{i0}e_{j0} = e_{i,-j}=-e_{j0}e_{i0}\quad (i>j\geq 1).
    \]
\end{lemma}

\begin{lemma}\label{l:R-A3-D3}
    %
    Точная обертывающая алгебра $RD_3(K)$ алгебры Ли $N\Phi(K)$ типа $A_3$ является
    неассоциативной и нестандартной.
\end{lemma}

\begin{proof}
    %
    Алгебры Ли $N\Phi(K)$ типа $A_3$ и $D_3$ изоморфны, поскольку их системы корней
    эквивалентны. Однако, алгебра $RD_3(K)$, в отличие от алгебры $RA_3(K)=
        NT(4,K)$, имеет нестандартный идеал
    \[
        K(e_{2 1}+ e_{2,-1})+ K(e_{3 1}+ e_{3,-1})+ Ke_{3,-2}
    \]
    и является неассоциативной, так как $e_{2,-1} (e_{32} e_{21}) = e_{3,-2}$ и
    $(e_{2,-1}e_{32}) e_{21}=0$.
\end{proof}

\section{Стандартные обертывающие алгебры типа $F_4$}
\begin{theorem}[{{\cite[теорема 1.1]{jsfu2018}}}]\label{th:stand-f4}
    %
    Существует стандартная обертывающая алгебра $R_\phi$ типа $F_4$.
    %Знаки структурных констант $N_{rs}$ алгебры $N\Phi(K)$ типа $F_4$ можно выбрать
    %так, что все идеалы кольца $R_\phi$ будут стандартными.
\end{theorem}
\begin{proof}
\end{proof}

\begin{figure}[ht]
    \centerfloat{
        \begin{picture}(200,450)

            %нижняя вертикальная полоса

            \put(55,10){$q_{4,-3}=p_{4,-4}$} \put(55, 20){\line(0,1){20}}
            \put(132,10){$_{(2342)}$}

            \put(55,50){$q_{4,-2}$} \put(55, 60){\line(0,1){20}}
            \put(132,50){$_{(1342)}$} %\put(160,50){$(1\leq \mid j \mid < i
            %\leq 4)$}

            \put(55,90){$q_{4,-1}$} \put(55, 100){\line(0,1){20}}
            \put(132,90){$_{(1242)}$} %\put(160,90){$\overline{p_{ij}}=q_{ij},\
            %\ \overline{q_{ij}}=p_{ij}$}

            \put(55,130){$p_{4,-3}=q_{40}$} \put(132,130){$_{(1232)}$}
            %++++++++++++++++++++++++++++++++++
            % ромб p(4,-3)p(4,-2)p(4,-1)q(41)
            \put(50,222){$p_{4,-1}$} \put(20,222){$_{(1221)}$}
            \put(0,175){$p_{4,-2}$} \put(50, 140){\line(-1,1){30}}
            \put(-27,175){$_{(1231)}$}

            \put(90,175){$q_{41}$} \put(60, 140){\line(1,1){30}}
            \put(110,175){$_{(1222)}$}

            \put(20, 185){\line(1,1){30}}

            \put(90, 185){\line(-1,1){30}}

            %+++++++++++++++++++++++++++++++++++++++
            % ромб р(4,-1)q(3,-2)q(3,-1)p(4,1)
            \put(50,310){$q_{3,-1}$} \put(20,310){$_{(1120)}$}
            \put(0,265){$q_{3,-2}$} \put(50, 230){\line(-1,1){30}}
            \put(-30,265){$_{(1220)}$}

            \put(90,265){$p_{41}$} \put(60, 230){\line(1,1){30}}
            \put(110,265){$_{(1121)}$}

            \put(20, 275){\line(1,1){30}}

            \put(90, 275){\line(-1,1){30}}

            %++++++++++++++++++++++++++++

            \put(50,222){$p_{4,-1}$} \put(20,222){$_{(1221)}$}

            \put(100, 185){\line(1,1){30}}

            \put(130,222){$q_{42}$} \put(150,222){$_{(1122)}$}
            %+++++++++++++++++++++++++++++=

            % параллелограмм q(41)p(4,-1)p(41)p(3,-2)p(3,-3)q(4,2)
            \put(130, 230){\line(-1,1){30}}

            \put(140, 230){\line(1,1){30}}

            \put(165,265){$p_{3,-3}=q_{43}$}

            \put(230,265){$_{(0122)}$}

            \put(100, 275){\line(1,1){30}}

            \put(170, 275){\line(-1,1){30}}

            \put(130,310){$p_{3,-2}$} \put(160,310){$_{(0121)}$}

            %++++++++++++++++++++++++++++++
            \put(90,310){$p_{42}$}

            \put(95, 275){\line(0,1){20}} \put(87,300){$_{(1111)}$}

            \put(90, 320){\line(-1,1){30}}

            \put(100, 320){\line(1,1){30}}

            %++++++++++++++++++++++++++++++++++++

            \put(55, 320){\line(0,1){30}}

            \put(135, 320){\line(0,1){30}}

            \put(20,355){$p_{43}=q_{30}$} \put(-5,355){$_{(1110)}$}

            \put(130,355){$p_{3,-1}$} \put(160,355){$_{(0111)}$}

            %++++++++++++++++++++++++++=====

            \put(50, 365){\line(-1,1){30}}

            \put(15,400){$q_{31}$}

            \put(-15,400){$_{(1100)}$}

            \put(60, 365){\line(1,1){30}}

            \put(20, 410){\line(1,1){30}}

            \put(90, 410){\line(-1,1){30}}

            \put(67,400){$q_{20}=p_{2,-1}$}

            \put(35,445){$q_{21}=p_{1,-1}$}

            %++++++++++++++++++++++++++++++

            \put(130, 365){\line(-1,1){30}}

            \put(170,400){$p_{31}$}

            \put(190,400){$_{(0011)}$}

            \put(140, 365){\line(1,1){30}}

            \put(100, 410){\line(1,1){30}}

            \put(170, 410){\line(-1,1){30}}

            %\put(67,400){$q_{20}=p_{2,-1}$}

            \put(120,445){$q_{10}=p_{21}$}

            %++++++++++++++++++++++++++=

            \put(-20,445){$q_{32}$}

            \put(15, 410){\line(-1,1){30}}

            %++++++++++++++++++

            \put(205,445){$p_{32}$}

            \put(180, 410){\line(1,1){30}}
            %+++++++++++++++++++++++++++++

            %\put(95, 365){\line(0,1){30}}
            \put(95, 360){\line(0,1){10}} %\put(95, 367){\line(0,1){10}}
            \put(95,374){\line(0,1){10}}
            %\put(95, 381){\line(0,1){10}}
            \put(95,388){\line(0,1){8}}

            %+++++++++++++++++++++++++
            \put(73,355){$_{q_{2,-1}=p_{2,-2}}$}

            %\put(100, 350){\line(1,-1){30}}
            \put(100, 350){\line(1,-1){10}} \put(112, 338)
            {\line(1,-1){10}}\put(124, 326){\line(1,-1){10}}

            %\put(90,350){\line(-1,-1){30}}
            \put(90,350){\line(-1,-1){10}}
            \put(78,338){\line(-1,-1){10}}
            \put(66,326){\line(-1,-1){10}}

        \end{picture}
    }
    \caption{Система положительных корней $F_4$}\label{fig:f4}
\end{figure}

\begin{table}[htbp]
    \begin{center}
        \begin{threeparttable}% выравнивание подписи по границам таблицы
            \caption{Значения $B(\Phi,m)$ для типов $F_4$ и $E_n$.}\label{table:Bmn}
            \begin{tabular}{ | c | c | c | c | c | c | c | c | c | c |}
                \hline
                $\Phi$/m & 0 & 1   & 2    & 3    & 4    & 5    & 6    & 7   & 8 \\ \hline
                % $G_2$ &  1 & 6  & 1 & & & & & &\\ \hline
                $F_4$    & 1 & 24  & 55   & 24   & 1    &      &      &     &   \\ \hline
                $E_6$    & 1 & 36  & 204  & 351  & 204  & 36   & 1    &     &   \\ \hline
                $E_7$    & 1 & 63  & 546  & 1470 & 1470 & 546  & 63   & 1   &   \\ \hline
                $E_8$    & 1 & 120 & 1540 & 6120 & 9518 & 6120 & 1540 & 120 & 1
                \\\hline
            \end{tabular}
        \end{threeparttable}
    \end{center}
\end{table}
Таблица~\cref{table:Bmn}, рисунок~\cref{fig:f4}.

\section{Теоремы единственности}\label{sec:uniqtheorems}
В этом разделе мы исследуем для классических типов $\Phi$ условия однозначности
обертывающих алгебр $R_{\Phi}$ над полем.

\medskip

Ясно, что для любой подсистемы $\Psi$ системы корней $\Phi$ точная обёртывающая
алгебры Ли $N\Psi(K)$ определена в алгебре $R_{\Phi}$ как подалгебра
\[
    R(\Psi)\coloneq\sum_{a \in \Psi^+}Ke_a.
\]

Когда в системе $\Phi$ все корни одной длины и база $\Pi(\Psi)$ подсистемы
$\Psi$ лежит в базе $\Pi(\Phi)$, нестандартный идеал подалгебры $R(\Psi)$
порождает в алгебре $R_{\Phi}$ также нестандартный идеал. Отсюда вытекает

\begin{lemma}\label{l:stand-subsystem}
    Пусть $\Pi(\Psi) \subseteq \Pi(\Phi)$ для подсистемы $\Psi$
    системы корней $\Phi$ одной длины. Если точная обёртывающая
    $R_{\Phi}$ алгебры Ли $N\Phi(K)$ стандартная, то подалгебра
    $R(\Psi)$ алгебры $R_{\Phi}$ также стандартна.
\end{lemma}

Как и в \cite{JA-12}, корни $r, s$ называем \emph{$p$-связанными} для простого
корня $p$, если $r+p$, $s+p$ $\in \Phi^+$. Подсистему корней $\Psi(r, p, s)$ в
$\Phi$ определяем равенствами
\[
    \Psi=\Psi(r, p, s)\coloneq\Phi\cap (Zr+Zp+Zs).
\]

Алгебры $R_{\Phi}$ типа $A_{2}$ ассоциативны, поскольку для них $R_{\Phi}^3=0$.
В системах корней $\Phi$ одной длины подсистема $\Psi$ ранга 3 всегда типа
$A_3$. Очевидна

\begin{lemma}\label{l:subsystem-rank3}
    Если подсистема корней $\Psi=\Psi(r,p,s)$ ранга $>2$, то либо она
    типа $A_3$ с базой $\{r, p, s\}$, либо $\Psi$ типа $B_3$ или $C_3$.
\end{lemma}

Замена $a\circ b\coloneq ba$ умножения в $R$ дает \emph{противоположную
    алгебру} $R^{(op)}$, антиизоморфную $R$.

\begin{lemma}\label{l:NT-opposite}
    Алгебра $NT(n,K)$ нижних нильтреугольных $n\times n$ матриц над полем $K$
    изоморфна своей противоположной алгебре $NT(n,K)^{(op)}$.
\end{lemma}
\begin{proof}
    Пусть $J=(j_{ab})$ "--- матрица перестановки с единицами на побочной диагонали,
    то есть $j_{a,n+1-a}=1$ и $j_{ab}=0$ иначе. Тогда $J^{-1}=J$.

    Рассмотрим линейное отображение
    \[
        \varphi\colon NT(n,K)\to NT(n,K),\qquad \varphi(X)=JX^{t}J^{-1}=JX^tJ,
    \]
    где $X^t$ "--- транспонированная матрица. Если $X$ нижнетреугольна, то $X^t$
    верхнетреугольна, а сопряжение матрицей $J$ переводит верхнетреугольные матрицы
    в нижнетреугольные; значит, $\varphi(NT(n,K)) = NT(n,K)$.

    Для любых $X,Y\in NT(n,K)$ имеем
    \[
        \varphi(XY)=J(XY)^tJ=JY^tX^tJ=(JY^tJ)(JX^tJ)=\varphi(Y)\varphi(X),
    \]
    то есть $\varphi$ "--- антиавтоморфизм алгебры $NT(n,K)$. Следовательно,
    отображение $X\mapsto \varphi(X)$ задает изоморфизм $NT(n,K)\simeq
        NT(n,K)^{(op)}$.
\end{proof}

\begin{lemma}\label{l:stand-opposite}
    Если точная обертывающая алгебра $R$ алгебры Ли $N\Phi(K)$ стандартна, то
    $R^{(op)}$ также является стандартной обертывающей алгеброй.
\end{lemma}
\begin{proof}
    Очевидно, что множества идеалов в $R$ и в $R^{(op)}$ совпадают.
    Эндоморфизм $x \mapsto -x \ (x \in R)$ модулей $R$ и $R^{(-)}$ является антиавтоморфизмом
    алгебры $R^{(-)}$ в силу равенств
    \[
        [-a, -b] = [a, b] = ab - ba = -[b, a] \ (a,b \in R).
    \]
    Его композиция $-\mu\colon x\mapsto -x^{\mu}$ с любым антиизоморфизмом $\mu$
    алгебры $R$ дает изоморфизм алгебр $R^{(-)}$ и $\mu (R)^{(-)}$. (В важном
    частном случае этот изоморфизм хорошо известен \cite[Раздел 11.2.1]{carter72}.)

    Остается заметить, что алгебры $R$ и $R^{(op)}$ определены на одном множестве и
    их антиизоморфизмом является тождественное отображение.
\end{proof}

Отметим, что переход к противоположной алгебре $R^{(op)}$ сохраняет и
стандартность, и ассоциативность.

\begin{lemma}\label{l:stand-A3}
    Пусть $\Psi=\Psi(r, p, s)$ "--- подсистема типа $A_3$ для
    $p$-связанных корней $r, s$ в $\Phi$. Тогда стандартность
    подалгебры $R(\Psi)$ в $R_{\Phi}$ равносильна условию $N_{s,p} = N_{p, r}$ и
    дает равенство $N_{s+p,r}=N_{s,r+p}$, причем
    $R(\Psi) \simeq NT(4,K)$, когда $N_{s+p,r} =1$. Если $N_{s+p,r}= -1$, то
    $R(\Psi)$ "--- неассоциативная алгебра.
\end{lemma}

\begin{proof}
    В подсистеме $\Psi=\Psi(r,p,s)$ типа $A_3$ корень $r+p+s$ "--- единственный,
    представимый неоднозначно суммой специальной пары корней.

    Любую обертывающую алгебру $R(\Psi)$ алгебры Ли $N\Psi(K)$ порождают $Ke_r$,
    $Ke_s$ и $Ke_p$. Ненулевые структурные константы для $R(\Psi)$ типа $A_3$ равны
    $\pm 1$.

    Для любого кольца $A$ наряду с его односторонними аннуляторами выделяют
    аннулятор
    \[
        \Ann\ (A)= \{\alpha\in A\ | \ \alpha A=0= A\alpha\}.
    \]
    Аннулятор кольца $R(\Psi)$ есть идеал $Ke_{r+p+s}=R(\Psi)^3$, причем
    \[
        R(\Psi)=Ke_r+(T(p)\cap R(\Psi))+Ke_s.
    \]
    Пересечение $T(p)\cap R(\Psi)$ и его аннулятор в $R(\Psi)$ совпадают с
    аннулятором идеала $R(\Psi)^2$. Если по модулю $ R(\Psi)^3$ односторонний
    аннулятор пересечения (равносильно, элемента $e_p$) совпадает с $R(\Psi)$ и,
    например, $e_p e_r=e_{r+p},\ e_re_p=0$, то $K(e_p+ e_s)$ порождает в $R(\Psi)$
    нестандартный идеал, как и в лемме~\ref{l:R-A3-D3}.

    Таким образом, при условии стандартности $R(\Psi)$ элементы $e_r, e_s$ лежат в
    разных односторонних аннулляторах элемента $e_p$. Это означает, что совпадают
    знаки констант $N_{p,r}$ и $N_{s,p}$, равносильно, $N_{p,r} N_{s,p}=1$.

    С учетом леммы \ref{l:stand-opposite}, с точностью до перехода к
    противоположной алгебре, имеем
    \begin{equation}\label{eq:An-Nsp-Npr-1}
        N_{s,p} =N_{p,r}=1,\quad
        e_se_p=e_{s+p},\quad
        e_pe_r=e_{r+p},\quad
        e_re_p=0=e_pe_s.
    \end{equation}

    Тождество Якоби в кольце Ли $R(\Psi)^{(-)}$ дает равенства
    \[
        [e_s,[e_p,e_r]] = [[e_s,e_p],e_r],\quad N_{pr}N_{s,r+p}=N_{sp}N_{s+p,r}.
    \]
    Отсюда $N_{s,r+p}=N_{s+p,r}\ (=\pm 1)$, то есть $N_{s,r+p}N_{s+p,r} = 1$. В
    случае
    \begin{equation}\label{eq:An-Nspr-1}
        N_{s+p,r} = 1, \quad
        e_{s+p}e_r=e_{r+s+p}=e_{s}e_{r+p}, \quad
        e_re_{s+p}=0=e_{r+p}e_s,
    \end{equation}
    приходим к изоморфизму алгебр $R(\Psi)\simeq NT(4,K)$, действующему по правилу:
    \[
        e_p \mapsto e_{32}, \ \ e_r \mapsto e_{21}, \ \ e_s \mapsto
        e_{43}, \ \ e_{r+p} \mapsto e_{31}, \ \ e_{s+p} \mapsto e_{42}, \
        \ e_{s+r+p } \mapsto e_{41}.
    \]

    В оставшемся случае, когда $N_{s+p,r} = -1$, имеем $e_p(e_re_s)=0$ и
    $(e_pe_r)e_s = e_{s+r+p}$. Поэтому алгебра $R(\Psi)$ неассоциативная; по модулю
    аннуллятора она изоморфна алгебре $NT(4,K)/Ke_{41}$. Лемма доказана.
\end{proof}

\begin{lemma}\label{l:stand-AnMod3}
    Если точная обертывающая алгебра $R$ алгебры Ли $N\Phi(K)$ типа $A_n$
    $(n>2)$ стандартна, то по модулю третьих степеней алгебр имеем
    \[
        R\simeq NT(n+1, K)\quad {\mbox {или } }\ \ R^{(op)}\simeq NT(n+1, K).
    \]
\end{lemma}
\begin{proof}
    Исследуем произвольную стандартную обертывающую алгебру $R$
    алгебры Ли $N\Phi(K)$ типа $A_{n}\ (n> 2)$. Для любой подсистемы
    корней $\Psi= \Psi(r, p,s)$ типа $A_3$ в $\Phi$, по лемме
    \ref{l:stand-subsystem}, свойство стандартности $R$ наследуется
    подалгеброй $R(\Psi)$ "--- обертывающей алгебры Ли $N\Psi(K)$.

    \medskip

    Ясно, что пара $p$-связанных корней $r,s\in \Phi^+$ с простым корнем $p$
    существует только если $p$ "--- промежуточный корень в графе Кокстера
    \[
        \begin{picture}(300,30)
            {\footnotesize \put(-20,15){$A_{n}:$}}
            \put(255,15){ $(n$ вершин).}
            %,\$n\geqslant1$

            \put(10,15){\circle{6}} \put(191,15){\line(1,0){50}}
            \put(244,15){\circle{6}} \put(188,15){\circle{6}}
            \put(69,15){\line(1,0){50}} \put(122,15){\circle{6}}
            \put(66,15){\circle{6}} \put(253,15)
            % {.}

            \put(85,15){\line(1,0){10}}

            \put(125,15){\line(1,0){7}} \put(137,15){\line(1,0){6}}
            \put(148,15){\line(1,0){6}} \put(159,15){\line(1,0){6}}
            \put(170,15){\line(1,0){6}} \put(180,15){\line(1,0){5}}
            \put(13,15){\line(1,0){50}}
        \end{picture}
    \]
    \noindent Вершины графа Кокстера системы корней $\Phi$ соответствуют простым
    корням. Ясно, что специальные пары простых корней
    экстраспециальны.

    Пусть $\Psi=\Psi(r, p,s)$ "--- подсистема в $\Phi$ с базой из простых корней
    $r,p,s$. Учитывая леммы \ref{l:stand-one-corner} и \ref{l:stand-A3}, для
    стандартной алгебры $R(\Psi)$ элементы $e_r, e_s$ лежат в разных односторонних
    аннулляторах элемента $e_p$ "--- в левом $\Ann^{(l)} (e_p)$ и правом
    $\Ann^{(r)}(e_p)$. В этом случае знаки структурных констант $N_{s,p}$ и
    $N_{p,r}$ совпадают и $N_{s,p}N_{p,r}=1$.

    Когда корень $s$ соответствует последней вершине графа Кокстера, в силу леммы
    \ref{l:stand-opposite}, с точностью до перехода от $R$ к противоположной
    алгебре, получаем соотношения \eqref{eq:An-Nsp-Npr-1}.

    \medskip

    Выберем далее корень $q$, соседний слева с $r$ в графе Кокстера, и подсистему
    $\Psi=\Psi (q, r, p)$ с базой $q, r, p$. Применяя леммы
    \ref{l:stand-one-corner} и \ref{l:stand-A3} к подалгебре $R(\Psi)$, аналогично
    получаем $e_re_q = e_{r+q}$.

    Указанный процесс продолжаем, завершая подалгеброй $R(\Psi)$ с базой в $\Psi$,
    начинающейся с первой вершины графа Кокстера.

    Базу обертывающей алгебры $R=NT(n+1, K)$ и базу Шевалле алгебры Ли~$R^{(-)}$
    дают матричные единицы $e_r= e_{ij}$ $(1 \leq j < i \leq n+1)$ при
    соответствующей нумерации корней $r=r_{ij}$ системы $\Phi$ типа $A_{n}$.

    В матричной индексации и обозначениях $e_{r}=e_{ij}$ при $r=r_{ij}$ получаем
    $e_{ij}e_{jm}=e_{im}$ для случаев $i-m=2$. Это дает изоморфизм алгебр $R$ и
    $NT(n+1,K)$ по модулю их третьих степеней. Лемма доказана.
\end{proof}

Все стандартные и все ассоциативные точные обертывающие алгебры $R$ алгебры Ли
$N\Phi(K)$ типа $A_{n-1}\ (n> 3)$ над полем $K$ классифицирует

\begin{theorem}[{{\cite[теорема 1]{imm2020}}}]\label{th:Stand-An}
    Точная обертывающая алгебра $R$ алгебры Ли $N\Phi(K)$ типа
    $A_{n-1}$ стандартна тогда и только тогда, когда алгебра $R$ или
    $R^{(op)}$ по модулю аннулятора изоморфна $NT(n,K)/Ke_{n1}$.
    Ассоциативная точная обертывающая алгебра алгебры Ли $N\Phi(K)$
    типа $A_{n-1}$, с точностью до изоморфизма
    единственна и изоморфна алгебре $NT(n,K)$.
\end{theorem}

\begin{proof}
    Исследуем произвольную стандартную обертывающую алгебру $R$
    алгебры Ли $N\Phi(K)$ типа $A_{n}\ (n> 2)$. Для $n=3$ теорему
    \ref{th:Stand-An} доказывают леммы \ref{l:stand-A3} и \ref{l:stand-AnMod3}; в этом случае
    $\Phi=\Psi=\Psi(r, p, s)$ типа $A_3$, а пара $p$-связанных корней $r,s$ в
    $\Phi^+$ и простой корень $p$ определены однозначно.

    Каждая подалгебра $R(\Psi)$ в $R$ с подсистемой корней $\Psi= \Psi(r, p,s)$
    типа $A_3$ в $\Phi$ стандартна в силу леммы \ref{l:stand-subsystem}. С
    точностью до перехода от $R$ к противоположной алгебре, по лемме
    \ref{l:stand-A3} имеем соотношения \eqref{eq:An-Nsp-Npr-1}, а в случае
    ассоциативности $R$ также \eqref{eq:An-Nspr-1}. При $n > 3$ по индукции можем
    считать теорему доказанной для каждого типа $A_{k}$, $3\leq k <n$.

    Выберем в $\Phi$ подсистему $\Phi_{1}$ (аналогично $\Phi_{n}$) с базой,
    полученной из $\Pi$ отбрасыванием первого (соответственно, последнего) простого
    корня в графе Кокстера; обе подсистемы типа $A_{n-1}$. Ясно, что стандартность
    алгебры $R$ наследуется ее подалгебрами $R(\Phi_{1})$ и $R(\Phi_{n})$, по
    индуктивному предположению, и дает их ассоциативность, а также изоморфность
    алгебр $R$ и $NT(n+1,K)$ по модулю $n$-х степеней.

    Стандартность подалгебр $R(\Psi)$ в $R$ для подсистем корней $\Psi = \Psi(r,
        p,s)$ типа $A_3$ в $\Phi_{l} \cup \Phi_{r}$ приводит к уточнению. Матричная
    индексация корней и обозначения $e_{r}=e_{ij}$ при $r=r_{ij}$ $(1\leq j<i\leq
        n)$, наряду с ассоциативностью произведения $e_{n+1 n}e_{n n-1} \cdots
        e_{32}e_{21}=e_{n+1 1}$ и подалгебр $R(\Psi(r_{j1}, r_{ ij}, r_{n+1 i}))$ при
    $1< j<i< n+1$, по лемме \ref{l:stand-A3}, дают также изоморфность алгебр $R$ и
    $NT(n+1,K)$.

    Это завершает доказательство теоремы.
\end{proof}

\begin{remark}\label{remark:Stand-Non-Assoc-An}
    Замена соотношений $e_{nj}e_{j1}= e_{n1}$ и $e_{j1}e_{nj}=0$ новыми
    \[
        e_{nj}e_{j1}=0,\quad e_{j1}e_{nj}=e_{n1}\qquad (j=2,3,\ldots,n-1)
    \]
    приводит алгебру $NT(n,K)\ (n>3)$ к неассоциативной стандартной обертывающей
    алгебре типа $A_{n-1}$; обе алгебры по модулю аннулятора $Ke_{n1}$ изоморфны.
    Поэтому требование стандартности точной обертывающей алгебры здесь слабее
    условия ассоциативности, с учетом теоремы \ref{th:Stand-An}.
\end{remark}

\medskip

Перечислим возможные типы $\Phi$ стандартных обертывающих алгебр $R_{\Phi}$.

\begin{proposition}\label{p:Exc-Dn-En}
    Стандартная обертывающая $R_{\Phi}$ алгебры Ли $N\Phi(K)$
    существует для всех типов ${\Phi}$, кроме типа $D_n$ $(n \geq 4)$
    и $E_n\ (n=6,7$ и $8)$.
\end{proposition}

\begin{proof}
    Существование стандартной обертывающей алгебры $R_{\Phi}$ над
    полем для $\Phi$ типа $B_n$ и $C_n$ указывает (\cite[Теорема
        4]{DAN2018}, \cite{imm2020}) следующая

    \begin{theorem}\label{th:classical-Exc-RDn}
        Пусть $R_{\Phi}$ есть алгебра классического типа из списка
        \ref{eq:RAn-RDn}. Если $s$ "--- угол идеала $H$ кольца $R_{\Phi}$ и
        $Q(s) \not\subseteq H$, то $\Phi$ типа $D_n$, $s=r_{iv}$, $2 \leq
            i<n$, $v=\pm 1$, причем идеалы $T(r_{i+1,1})$ и $T(r_{i+1,-1})$
        оба не лежат в $H$.
    \end{theorem}

    В силу теорем \ref{th:Stand-An} и \ref{th:classical-Exc-RDn}, для алгебр Ли
    $N\Phi(K)$ классических типов доказательство предложения \ref{p:Exc-Dn-En}
    требуется лишь для типа $D_n, \ n>3$.

    \medskip

    Допустим, что стандартная точная обертывающая алгебра $R_{\Phi}$ для $\Phi$
    типа $D_n$ существует. Граф Кокстера системы корней $\Phi$ (его вершины
    соответствуют простым корням) типа $D_n$ представляется в виде

    \medskip

    \noindent
    \[
        \begin{picture}(300,30)
            {\footnotesize \put(-20,15){$D_n:$}} \put(255,15) { $(n$ вершин).
                % ,\ $n\geqslant 4)$
            } \put(198,15){\circle{6}}\put(117,15){\line(1,0){40}}
            \put(113,15){\circle{6}}\put(244,15){\circle{6}}
            \put(201,15){\line(1,0){40}} \put(66,15){\circle{6}}
            \put(69,15){\line(1,0){40}} \put(163,15){\line(1,0){6}}
            \put(172,15){\line(1,0){5}} \put(181,15){\line(1,0){5}}
            \put(189,15){\line(1,0){6}} \put(115,15){\line(1,0){8}}
            \put(160,15){\circle{6}} \put(63,15.5){\line(-5,1){50}}
            \put(63,14.5){\line(-5,-1){50}} \put(10,26){\circle{6}}
            \put(10,4){\circle{6}}
        \end{picture}
    \]

    \noindent
    %
    Нетривиальная симметрия \ $\bar {}$ \ графа Кокстера единственна при $n>4$. Она
    переставляет крайние слева вершины $r_{2, -1}$ и $r_{2 1}$; остальные вершины
    неподвижны. Если $r\neq {\bar {r}}$ для линейного продолжения \ $\bar {}$ \
    перестановки базы $\Pi=\Pi (\Phi)$ на $\Phi$, то $r=r_{iv}$ и ${\bar {r}}=
        r_{i, -v}$ при одном из двух значений $v=\pm 1$.

    Обозначим через $\Phi_v$ подсистему корней типа $A_{n-1}$ в $\Phi$ с базой
    %
    \[\Pi_v=\Pi (\Phi)\setminus \{r_{2, -v}\}, \ v=\pm 1.\]
    %
    Тогда в $N\Phi (K)$ находим подалгебру $N\Phi_v(K)$ с базой $\{ e_r \mid r\in
        \Pi_v\}$ типа $A_{n-1}$, как показывают графы Кокстера. По лемме
    \ref{l:stand-subsystem} получаем стандартность точных обертывающих алгебр
    $R(\Phi_v)$, $v=\pm 1$, как подалгебр в $R_{\Phi}$.

    В алгебрах $R(\Phi_1)$ и $R(\Phi_{-1})$ стандартна и каждая подалгебра
    $R({\Psi})$ для подсистемы $\Psi=\Psi(r,p,s)$ типа $A_{3}$ в $\Phi$ с условием
    $\{r ,p,s\}\subseteq \Pi_{1}$ или $\{r ,p,s\}\subseteq \Pi_{-1}$. Две такие
    подалгебры $R({\Psi})$ получаем для случаев $r=r_{2, -1}$ и $r=r_{21}$. Для них
    элементы $e_r$ и $e_s$ лежат в разных односторонних аннуляторах элемента
    $e_p=e_{32}$; соответственно, $R(\Psi)$ есть левый или правый аннулятор
    элемента $e_{32}$. Поэтому для $\Psi_0 =\Psi_0 \{r_{2, -1}, r_{32}, r_{2 1}\}$
    подалгебра $R({\Psi_0})$ типа $A_{3}$ в $R_{\Phi}$, в силу леммы
    \ref{l:stand-A3}, не является стандартной, как и алгебра $R_{\Phi}$.

    Полученное противоречие доказывает предложение \ref{p:Exc-Dn-En} для типа
    $D_n$.

    \medskip

    Графы Кокстера систем корней типа $E_n \ (n=6,7,8)$ содержат, как подграф, граф
    Кокстера системы корней типа $D_4$. Поэтому здесь предложение \ref{p:Exc-Dn-En}
    получаем, как несложное следствие. Существование стандартной обертывающей
    алгебры $R_{\Phi}$ для $\Phi$ типа $F_4$ устанавливает теорема
    \ref{th:stand-f4}.

    Завершая доказательство для исключительных типов, отметим, что для $\Phi$ ранга
    2 предложение \ref{p:Exc-Dn-En} следует из леммы \ref{l:stand-one-corner}.
\end{proof}

\medskip

Исследуем условия однозначности обертывающих алгебр $R_{\Phi}$ типов $B_n$ и
$C_n$ над полем.

Системы корней $\Phi$ типа $B_n$ и $C_n$ дуальны друг другу. В терминологии
Ж.-П.~Серра \cite{Serre}, графы Кокстера у них совпадают, см. также замечание в
\cite[\S\ 1]{JA-12}. Однако, при $n>2$ различаются схемы Дынкина, то есть графы
Кокстера с приписанным каждой вершине числом (обычно, квадрат длины
соответствующего вершине корня, когда короткие корни длины 1),

\medskip

\noindent
%
\[
    \begin{picture}(300,30)
        \put(10,15){\circle{6}} \put(191,15){\line(1,0){50}}
        \put(244,15){\circle{6}} \put(188,15){\circle{6}}
        \put(69,15){\line(1,0){50}} \put(122,15){\circle{6}}
        \put(66,15){\circle{6}} \put(85,15){\line(1,0){10}}
        \put(125,15){\line(1,0){7}} \put(137,15){\line(1,0){6}}
        \put(148,15){\line(1,0){6}} \put(159,15){\line(1,0){6}}
        \put(170,15){\line(1,0){6}} \put(180,15){\line(1,0){5}}
        \put(10,19){\line(1,0){56}} \put(10,12){\line(1,0){56}}
        {\footnotesize  \put(-20,15){$B_n:$}} \put(255,15)

        \put (7,20){1} \put (63,20){2} \put (119,20){2} \put (241,20){2}
        \put (185,20){2}
    \end{picture}
\]
\noindent
\[
    \begin{picture}(300,30)
        \put(10,15){\circle{6}} \put(191,15){\line(1,0){50}}
        \put(244,15){\circle{6}} \put(188,15){\circle{6}}
        \put(69,15){\line(1,0){50}} \put(122,15){\circle{6}}
        \put(66,15){\circle{6}} \put(85,15){\line(1,0){10}}

        \put(125,15){\line(1,0){7}} \put(137,15){\line(1,0){6}}
        \put(148,15){\line(1,0){6}} \put(159,15){\line(1,0){6}}
        \put(170,15){\line(1,0){6}} \put(180,15){\line(1,0){5}}
        \put(10,19){\line(1,0){56}} \put(10,12){\line(1,0){56}}
        {\footnotesize \put(-20,15){$C_n:$}} \put(255,15)

        \put (7,20){2}

        \put (63,20){1}

        \put (119,20){1}

        \put (241,20){1}

        \put (185,20){1}

    \end{picture}
\]

\medskip

В отличие от типа $D_n$, в системах ${\Phi}$ типа $B_n$ и $C_n$ корни $r_{jv}$
и $r_{j, -v}$ всегда инцидентны, то есть в $R_{\Phi}$ один из идеалов
$T(r_{jv})$ и $T(r_{j, -v})$ лежит в другом.

\medskip

Далее идеал $T(r)$ обертывающей алгебры $R_{\Phi}$ классического типа
обозначаем при $r=r_{iv}$ через $T_{iv}$, кроме случая $v=1$ для типа $D_n$,
где
\[
    T_{i1}\coloneq T_{i,-1}+T(r_{i1}).
\]
Как показывают леммы \ref{l:stand-one-corner} и \ref{l:StructConst-vl90a_ru},
стандартны все идеалы алгебры $R_{\Phi}$, лежащие в идеале вида $T_{i,-1}$ или
"--- для типа $B_n$ "--- в $T_{1 0}$.

Учитывая лемму~\ref{l:ReductBnCnDn} и теорему~\ref{th:Stand-An}, на алгебры
$R_{\Phi}$ накладываем условие

\begin{equation}\label{IsomR-Bn}
    B_n:\ R_{\Phi}/T_{2,-1}\simeq NT(n+1,K);
\end{equation}
\begin{equation}\label{IsomR-Cn}
    C_n:\ R_{\Phi}/T_{2,-2}\simeq NT(n+1,K).
\end{equation}

Тогда элементы $e_{jm}$ алгебры $R_{\Phi}$ при $j>m>0$ умножаются по обычным
для матричных единиц правилам, причем для ${\Phi}$ типа $B_n$, в силу
предложения~\ref{p:mult-enveloping},

\begin{equation}\label{IsomR-B3}
    e_{jm}e_{m0}=e_{j0},\qquad e_{m0}e_{j0}=- e_{j0}e_{m0}=c_{jm}
    e_{j,-m}\ \ (c_{jm}=\pm 1).
\end{equation}

\begin{lemma}\label{Subsistem-B3}
    Алгебра $R_{\Phi}$ типа $B_3$ с условием (\ref{IsomR-Bn})
    стандартная. С точностью до выбора знака произведения $e_{10}e_{2
                0}=\pm e_{2,-1}$, она определена однозначно, когда подалгебра
    $R(\Psi)$ для подсистемы $\Psi$ типа $A_3$ в $\Phi$ ассоциативная.
\end{lemma}

\begin{proof}
    Все длинные корни системы корней $\Phi$ типа $B_3$ образуют
    единственную подсистему $\Psi$ типа $A_3$. С учетом равенств
    (\ref{IsomR-B3}), условие (\ref{IsomR-Bn}) легко дает
    стандартность алгебры $R_{\Phi}$.

    Базу в $\Psi$ дают корень $r_{32}$ и $r_{32}$-связанные корни $r_{2,-1}$ и
    $r_{21}$, то есть
    \[
        \Psi= \Psi(r_{2,-1}, r_{32}, r_{21}).
    \]
    В силу (\ref{IsomR-Bn}), имеем $e_{32} e_{21}= e_{31}$, $e_{21} e_{32} = 0$.
    Поэтому стандартность подалгебры $R(\Psi)$ по лемме \ref{l:stand-A3}
    равносильна равенствам
    \[
        e_{32}e_{2, - 1}=0,\ \ e_{2,-1}e_{32} = e_{3,-1}.
    \]
    Тождество Якоби алгебры Ли $N{\Phi}(K)$, предложение~\ref{p:mult-enveloping} и
    лемма~\ref{l:StructConst-vl90a_ru} дают:
    %
    \[ [e_{20},e_{10}]=[[e_{21},e_{10}],e_{10}]=2ce_{2,-1},\quad
        e_{20}e_{10}=ce_{2,-1}\quad (c=\pm 1);\]
    %
    \[ [e_{30},e_{10}]=[[e_{32},e_{20}],e_{10}]=[e_{32},[e_{20},e_{10}]]=
        2ce_{3,-1},\ \ e_{30}e_{10}=ce_{3,-1}.\]
    %
    \[ [e_{30},e_{20}]=[e_{30},[e_{21},e_{10}]]=[e_{21},[e_{30},e_{10}]]=
        2ce_{3,-2},\ \ e_{30}e_{20}=ce_{3,-2}.\]
    %
    (См. также замечание \ref{remark:Stand-Non-Assoc-An}.) Далее,
    %
    \[[e_{2,-1},e_{31}]=[e_{2,-1},[e_{32},e_{21}]]=[[e_{2,-1},e_{32}],e_{21}]=[e_{3,-1},e_{21}]
        %
        =\pm e_{3,-2},\]
    %
    то есть либо $e_{2,-1}e_{31} = e_{3,-2}$ и $e_{31}e_{2,-1} = 0$, либо
    $e_{2,-1}e_{31} = 0$ и $e_{31}e_{2,-1} = e_{3,-2}$.
    %
    К ассоциативности подалгебры $R(\Psi)$ приводят оба случая:
    %
    \[e_{2,-1}(e_{32}e_{21})=e_{2,-1}e_{31}=e_{3,-2}=
        (e_{2,-1}e_{32})e_{21}=e_{3,-1}e_{21}. \]
    %
    Однако, выбор знака константы $N_{rs}$ для специальной пары $(r,s)$ корней с
    суммой $r+s= r_{3,-2}$ произволен, согласно ~\cite[Предложение
        4.2.2]{carter72}, лишь при условии экстраспециальности пары $(r,s)$.

    Таким образом, выбор знака $c=\pm 1$ определяет алгебру $R_{\Phi}$ однозначно.
\end{proof}

\begin{lemma}\label{Subsistem-Bn}
    Пусть $R_n=R_{\Phi}$ есть стандартная обертывающая алгебра типа
    $B_n$ $(n > 3)$ с условием $($\ref{IsomR-Bn}$)$. Тогда $R_{\Phi}=RB_n
        (K)$ по модулю идеала $T_{n,-1}$, с точностью до выбора знака
    $c=\pm 1$ произведения $e_{20} e_{10}= ce_{2,-1}$.
\end{lemma}

\begin{proof}
    В системе корней ${\Phi}$ типа $B_n$ $(n>3)$ выделим подсистемы
    $\Psi_{m,j,i,k}$ типа $B_4$ с базой $\{r_{m0}, r_{jm}, r_{ij},
        r_{ki}\}$. Корень $r_{ij}$ и $r_{ij}$-связанные корни $r_{j,-m}$,
    $r_{ki}$ образуют базу подсистемы $\Psi$ типа $A_3$, то есть
    %
    \begin{equation}\label{SubsystA3-Bn}
        \Psi=\Psi (r_{j, -m}, r_{ij},  r_{ki}),\quad 1\leq m<j<i<k\leq n.
    \end{equation}
    %

    Пусть стандартная алгебра $R_{\Phi}$ выбрана с условием (\ref{IsomR-Bn}). По
    лемме \ref{l:stand-A3}, если $e_{ij} e_{j,-m} =0$ (как и в лемме
    \ref{Subsistem-B3} при $n=3$), то в подалгебре $R(\Psi)$, а при $i=j+1$ и в
    алгебре $R_{\Phi}$, находим нестандартные идеалы с двумя углами $r_{j,-m}$,\
    $r_{ki}$. Следовательно, $e_{ij}e_{j, -m}=e_{i,-m}$ и, по лемме
    \ref{l:stand-A3},
    %
    \[e_{j v}e_{ij}=0,\quad e_{ij} e_{j v}=e_{iv}=[e_{ij}, e_{j v}]\qquad  (i>j>|v|=m>0).\]
    %

    Корень $r_{2.-1}$ представляется суммой $r_{2.-1}=r_{10}+r_{20}$ специальной
    пары корней однозначно, причем
    %
    \[ [e_{20},e_{10}]=2ce_{2,-1},\quad e_{20}e_{10}=ce_{2,-1}\quad (c=\pm 1),\]
    %
    \[ [e_{j0},e_{10}]=[[e_{j2},e_{20}], e_{10}]=[e_{j2},[e_{20},e_{10}]]=
        2ce_{j,-1},\ \ e_{j0}e_{10}=ce_{j,-1}.\]
    %
    Для корней $r_{i,-j}$ при $1< j<i$ также имеем
    %
    \[ [e_{i0},e_{j0}]=[e_{i0},[e_{j1},e_{10}]]=[e_{j1},[e_{i0},e_{10}]]=
        2ce_{i,-j},\ \ e_{i0}e_{j0}=ce_{i,-j}.\]
    %

    Корни $r_{i,-m}$ и $r_{jm}$ инцидентны в системе $\Psi_{m,j,i,k}$, однако, в ее
    подсистеме (типа $D_4$) длинных корней не инцидентны. Вместе с $r_{kj}$ они
    образуют базу подсистемы $\Psi=\Psi(r_{i,-m}, r_{jm}, r_{kj})$ типа $A_3$.
    Подалгебра $R(\Psi)$ в $R_{\Phi}$ здесь также стандартная; в противном случае
    нестандартный идеал с двумя углами $r_{i,-m}$ и $r_{ki}$ находим в $R(\Psi)$ и
    в алгебре $R_{\Phi}$. Поэтому $e_{i,-m}e_{jm}=0$ и $e_{jm}e_{i,-m}=e_{i,-j}$.

    Используя лемму \ref{l:stand-A3} и тождество Якоби алгебры Ли $N{\Phi}(K)$,
    получаем
    \[
        e_{i,-j}=[e_{jm},e_{i,-m}]=[e_{jm},[e_{ij},e_{j,-m}]]= [e_{j,-m},
        [e_{ij}, e_{jm}]]=[e_{j,-m},e_{im}],
    \]
    %
    \begin{equation}\label{RelatA3-RBn}
        e_{i,-j}=e_{jm}e_{i,-m}=[e_{jm},e_{i,-m}]=[e_{j,-m},e_{im}]=e_{j,-m}e_{im}.
    \end{equation}
    %

    Учитывая произвол в выборе подсистемы $\Psi_{m,j,i,k}\ (i<k\leq n)$, с
    точностью до выбора знака $c=\pm 1$, имеем $R_{\Phi}=RB_n(K)$ по модулю идеала
    $T_{n,-1}$.
\end{proof}

\medskip

Согласно доказательству предложения 12.2.3 в \cite{carter72} (см. также
\cite{vl90a_ru}), алгебра Ли $N\Phi (K)$ допускает графовый автоморфизм
$\theta$, когда граф Кокстера системы корней $\Phi$ допускает нетривиальную
симметрию и все корни в $\Phi$ одной длины. Симметрия линейно продолжается до
перестановки \ $\bar {}$ \ системы корней $\Phi$.

\medskip

Симметрия \ $\bar {}$ \ системы корней $\Phi$ типа $D_{n}$ при $n>4$
единственна и графовый автоморфизм $\theta$ алгебры Ли $N\Phi(K)$ порядка 2
определяется правилом
\[
    \theta : \ e_r \to e_{\bar {r}}\quad (r \in \Phi^+)
\]
\begin{lemma}\label{RB_n-RDn+1}
    Алгебра $RB_n(K)$ представляется в алгебре $RD_{n+1}(K)$
    централизатором графового автоморфизма $\theta$ порядка 2 алгебры
    Ли $RD_{n+1}(K)^{(-)}$.
\end{lemma}

\begin{proof}
    В представлении алгебры Ли $N\Phi(K)$ типа $D_{n+1}$ алгеброй Ли
    $RD_{n+1}(K)^{(-)}$ графовый автоморфизм $\theta$ переставляет в
    $D_{n+1}^+$-матрицах 1-й и (-1)-й столбцы (то есть $\theta (e_{i
            v}) =e_{i, -v}$ при $v=\pm 1$), не изменяя остальные столбцы.

    Автоморфизм $\theta$ централизует (или оставляет неподвижными) все элементы
    \begin{equation}\label{IsomRBn-Dn}
        f_{i-1,0}: = e_{i,-1}+e_{i 1}\ \ {1<i\leq n+1},\quad f_{i-1,\pm
                (v-1)}: =e_{i, \pm v}\ \ (1< v<i\leq n).
    \end{equation}
    Очевидно, централизатор графового автоморфизма $\theta$ совпадает с подалгеброй
    \[
        C(\theta)=\sum_{i=2}^{n+1} K(e_{i,-1}+e_{i 1})+\sum_{1<|v|<i\leq n+1} Ke_{iv}
    \]
    и является даже идеалом алгебры $RD_{n+1}(K)$.

    Выбранная нумерация элементов $f_{iv}$ позволяет считать их базой модуля
    $RB_n(K)$. Умножение в алгебре $RD_{n+1}(K)$, по лемме \ref{l:mult-RAn-RDn},
    превращает модуль в алгебру $RB_n(K)$ с базой Шевалле из элементов $f_{iv}$.

    Требуемый изоморфизм $C(\theta)\to RB_n (K)$ укажем явным действием на
    произвольную $D_{n+1}^+$-матрицу $\alpha$ из $C(\theta)$, по аналогии со
    скручивающим автоморфизмом группы Шевалле типа $D_{n+1}$ перед леммой 4 в
    \cite{vl90a_ru}. Вначале "склеиваем" \ 1-ый и (-1)-й столбцы в $\alpha$, считая
    его 0-м, а остальные столбцы не изменяем. Заменяя в новой матрице номер $i$
    каждой строки на $i-1$, а номер $\pm (v-1)$ столбца элементов из них на номер
    $\pm v\ (1<v < i\leq n)$, получаем изоморфный образ элемента $\alpha$.
    Равенства (\ref{IsomRBn-Dn}) определяют обратное изоморфное вложение.
\end{proof}

\medskip

Лемма~\ref{RB_n-RDn+1} и замечание~\ref{remark:A3-D3} приводят к возрастающей
последовательности с однозначно определенными изоморфными вложениями алгебр:

\begin{equation}\label{InclusRBn-Dn}
    RB_{n-1}(K)\subset RD_n(K)\subset RB_n(K)\subset
    RD_{n+1}(K)\subset \cdots ,\quad n=3,4,5,\dots \ .
\end{equation}

\begin{definition}
    Последовательность обертывающих алгебр $R_n$ одного
    классического типа (скажем, $B_n$) возрастающих рангов $n$ назовем \emph{монотонной},
    если каждый ее член изоморфно представляется подалгеброй следующего члена.
\end{definition}

\medskip

Соответствующий пример с естественными изоморфными вложениями дает
%
\begin{equation}\label{InclusRBn}
    RB_3(K)\subset RB_4(K)\subset \cdots \subset RB_n(K)\subset
    RB_{n+1}(K) \subset \cdots .
\end{equation}
%
Монотонность здесь, очевидно, нарушается (вместе с первым включением) при
замене $RB_3(K)$ на алгебру $R_{\Phi}$ типа $B_3$ из леммы \ref{Subsistem-B3}.

\medskip

Сейчас мы можем указать условия однозначности обертывающих алгебр $R_{\Phi}$
классических типов. Непосредственно из леммы \ref{Subsistem-Bn} вытекает

\begin{theorem}\label{th:Stand-Bn}
    Если последовательность $R_n= R_{\Phi}$ стандартных обертывающих алгебр типа
    $B_n$ $(n=3, 4, \cdots )$ с условием \eqref{IsomR-Bn} монотонная, то $R_n=RB_n
        (K)$, с точностью до выбора знака $c=\pm 1$ произведения $e_{20} e_{10}=
        ce_{2,-1}$.
\end{theorem}

Отметим, что наряду с последовательностями стандартных обертывающих алгебр типа
$B_n$, в силу \eqref{InclusRBn-Dn}, определены и монотонные последовательности
обертывающих алгебр типа $D_n$.

\medskip

\begin{remark}\label{remark:Stand-Cn}
    По аналогии с теоремой \ref{th:Stand-Bn} выделяется обертывающая
    алгебра $R_{\Phi}$ типа $C_n\ (n\geq 3)$ с условием
    (\ref{IsomR-Cn}) "--- алгебра $RC_n(K)$. Подчеркнем, что построения
    изоморфных вложений основаны на лемме о гомоморфизмах систем
    корней \cite[Лемма 7]{VLev-82MZ}. Так, алгебра $RC_n(K)$
    представляется централизатором в алгебре $RA_{2n-1}(K)=NT(2n,K)$
    графового автоморфизма $\theta$ порядка 2 алгебры Ли
    $RA_{2n-1}(K)^{(-)}$. Получаем:

    \begin{equation}\label{IsomRCn-A2n-1}
        RC_n(K)\subset NT(2n,K)\subset RC_{n+1}(K)\subset
        NT(2n+2,K)\subset \cdots ,\  n=3,4,\dots  .
    \end{equation}
\end{remark}

\FloatBarrier

\chapter{Проблемы перечисления идеалов нильтреугольных алгебр Шевалле}\label{ch:enumeration}
\section{Проблема перечисления стандартных идеалов} \label{sec:Problem-1}
Проблемы комбинаторного перечисления стандартных идеалов и всех идеалов алгебр
Ли $N\Phi(q) \coloneq N\Phi(GF(q))$ классических типов записаны в 2001 году
\cite[Проблемы 1 и 2]{sigsam2001}. Полное решение проблемы~1 дает следующая
теорема (анонсировалась в \cite{DAN2018}), использующая $q$-биномиальные
коэффициенты

\[
    {n \brack k}_q \coloneq\prod_{i=0}^{k-1}\frac{1-q^{n-i}}{1-q^{i+1}}.
\]

\begin{theorem}\label{th:Enum-St-Id}
    Число стандартных идеалов алгебры Ли $N\Phi(q)$ классического типа
    $\Phi$ лиева ранга $n$ равно
    %
    \begin{equation*}
        1+\sum_{m=1}^n B(\Phi,m) \sum_{t=1}^m
        \sum_{k=0}^{m-t} (-1)^k \binom{m}{k} {m-k \brack t}_q,
    \end{equation*}
    где $B(\Phi,m)=\binom{n}{m}^2$ для типов $B_n$ и $C_n$, а также
    \[
        B(A_{n-1},m) = \frac{1}{n}\binom{n}{m}\binom{n}{m+1}, \quad B(D_n,m) =
        \binom{n}{m}\left( \binom{n-1}{m} + \binom{n-2}{m-2} \right).
    \]
\end{theorem}
\begin{proof}
    Согласно~\cite{JA-12}, стандартный идеал $H$ в $N\Phi(K)$ характеризуют
    множество $\mathcal{L}=\mathcal{L}(H)$ его углов и \emph{фрейм}
    $\mathcal{F}(H)$, определяемый условиями
    %
    \[\mathcal{F} (H)\subseteq \sum_{s\in \mathcal{L}}Ke_{s},\quad \mathcal{F}(H)=H \mod
        Q(\mathcal{L}).\]
    %
    Таким образом, стандартность $H$ равносильна условию $H=Q(\mathcal{L}) +
        \mathcal{F}(H)$.

    Известные перечисления (\cite[Следствие 4.3]{vL74}, \cite[Теорема
        2.1.2]{Egorychev84}, \cite{Eg-Lev-1996}, \cite{eS05}) нормальных подгрупп групп
    $U\Phi(K)$ над полем $K$, инвариантных относительно диагональных автоморфизмов,
    редуцируются к перечислению стандартных идеалов $H$ колец Ли $N\Phi(K)$ с
    фреймом $\mathcal{F}(H) = \sum_{s\in \mathcal{L}} Ke_{s}$, то есть вида
    $T(\mathcal{L})$. Тем самым, они редуцируются к перечислению определенных путей
    в решетках, зависящих от выбора типа алгебры $N\Phi(K)$.

    \medskip

    Пусть $K^m$ "--- пространство строк длины $m$ над $K$. Подпространство $S$
    назовем \emph{координатно полным}, если для каждого номера $i$ $(1 \leq i \leq
        m)$ в $S$ существует элемент с ненулевой $i$-той координатой. Ясно, что в
    алгебре Ли $N\Phi(K)$ любой стандартный идеал $H$ с
    $\mathcal{L}=\mathcal{L}(H)$ порядка $m$ записывается в виде
    \begin{equation}\label{hls}
        H(\mathcal{L},S) = Q(\mathcal{L}) + \{ a_1e_{r_1} + a_2e_{r_2}
        + \dots + a_me_{r_m} \,|\, (a_1,a_2,\dots,a_m)\in S \}
    \end{equation}
    для однозначно определенного координатно полного подпространства $S$ в $K^m$.

    \medskip

    Число всех координатно полных $t$-мерных подпространств пространства $K^m$ при
    $K=GF(q)$ обозначим через $\tilde V(m,t,q)$, а через $B(\Phi,m)$ "--- число
    всех $m$-элементных множеств углов $\mathcal{L}$ в $\Phi^+$. Из биективности
    соответствия \eqref{hls} между стандартными идеалами и парами $(\mathcal{L},
        S)$ сразу же вытекает

    \medskip

    \begin{lemma}\label{l:Reduct-BV}
        Число стандартных идеалов алгебры Ли $N\Phi(q)$ ранга $n$ равно
        \[
            1+\sum\limits_{m=1}^{n}B(\Phi,m)\, \ \sum\limits_{t=1}^{m}{\widetilde V}(m,t,q).
        \]
    \end{lemma}

    В~\cite{vmj2015} задача \textbf{(A)} исследовалась для типа $A_n$. Числа
    $B(\Phi,m)=B(A_{n},m)$ вычислены ранее \cite[Теорема~2.1.2]{Egorychev84} (см.
    также \cite{vL74} и \cite{Tolasov-1977}):
    \[
        B(A_{n-1},m) = \frac{1}{n}\binom{n}{m}\binom{n}{m+1}.
    \]

    Для требуемых в лемме~\ref{l:Reduct-BV} чисел $\widetilde{V}(m, t, q)$
    в~\cite{vmj2015} найдена формула
    \[
        \widetilde{V}(m,t,q) =\qquad \mkern-18mu \mkern-18mu \sum\limits_{1=j_1<j_2<\ldots
            < j_{t}\leq m}  \!\! \frac{(q^t-1)^{m-j_{t}}}{(q-1)^{t-j_{t}}}\cdot
        \prod_{k=2}^{t-1}\left(\frac{q^k-1}{q-1}\right)^{j_{k+1}-j_k-1}\mkern-18mu
        \mkern-18mu \mkern-18mu \mkern-18mu \qquad \qquad .
    \]

    \noindent
    Применяя к ней разработанный \cite{Egorychev84} метод интегрального
    представления комбинаторных сумм, включающих $q$-биномиальные коэффициенты, Г.\:П.~Егорычев
    установил \cite[Леммы~3 и 4]{IzvIrkGU-2016} рекуррентное соотношение
    \begin{equation}\label{eq:vmt-rec}
        \widetilde{V}(m,t,q) = \sum_{k=0}^{m-t} (q^t -1)^k \times
        \widetilde{V}(m-1-k,t-1,q)
    \end{equation}
    и следующее утверждение.

    \begin{lemma}\label{l:vmt-comb}
        Справедлива следующая формула
        \begin{equation}
            \label{eq:vmt-comb}
            \widetilde{V}(m,t,q) = \sum_{k=0}^{m-t} (-1)^{m-t-k} q^k \binom{m-1}{t+k-1}
            {t+k-1 \brack k}_q.
        \end{equation}
    \end{lemma}

    \medskip

    В~\cite[Заключение]{IzvIrkGU-2016} высказывалась потребность
    алгебраически-комбинаторного доказательства формул \eqref{eq:vmt-rec} и
    \eqref{eq:vmt-comb}. Доказана

    \begin{lemma}\label{l:vmt-comb-1}
        Справедлива формула
        \begin{equation}
            \label{eq:vmt-comb-1}
            \widetilde{V}(m,t,q) = \sum_{k=0}^{m-t} (-1)^k \binom{m}{k}
            {m-k \brack t}_q.
        \end{equation}
    \end{lemma}

    \begin{proof}
        К формуле \eqref{eq:vmt-comb-1} приводят $q$-аналог ${n \brack
                    k}_q = {n \brack n-k}_q$ известной комбинаторной формулы и
        подстановка в \eqref{eq:vmt-comb} тождества
        %
        \[q^k{t+k-1 \brack k}_q = {t+k \brack t}_q - {t+k-1 \brack t}_q.\]
        %
        Докажем формулу \eqref{eq:vmt-comb-1} для числа $\widetilde{V}(m, t,q)$ всех
        координатно полных $t$-мерных подпространств в $K^m$ при $K=GF(q)$. Н.\,Д.
        Ходюня получает рекуррентную формулу \eqref{eq:vmt-rec} перенесением схемы
        перечисления канонических баз в лемме \ref{l:LieIdeal}.

        \medskip

        Известно, что число всех $t$-мерных подпространств в $K^m$ при $K=GF(q)$ равно
        ${m \brack t}_q$. Пусть $U_i$ ($1\leq i \leq m$) "--- подпространство
        размерности $m-1$ всех векторов в $K^m$ c нулевой $i$-й координатой. Тогда
        любое подпространство в $U_i$ не является координатно полным в $K^m$, а каждое
        подпространство в $K^m$, не являющееся координатно полным, лежит хотя бы в
        одном подпространстве $U_i$.

        \medskip

        Обозначим через $\hat{U}_i$ множество всех $t$-мерных подпространств в $U_i$.
        Формулу~\eqref{eq:vmt-comb-1} находим, пользуясь формулой включений-исключений,
        %
        \[
            \widetilde{V}(m,t,q) = {m \brack t}_q - \sum_i|\hat{U}_i| +
            \sum_{i \neq j}|\hat{U}_i \cap \hat{U}_j| - \sum_{i \neq j \neq
                k}|\hat{U}_i \cap \hat{U}_j \cap \hat{U}_k| + \cdots =
        \]
        \[
            {m \brack t}_q-\binom{m}{1}{m-1 \brack t}_q + \binom{m}{2}{m-2
                \brack t}_q
            + \cdots + (-1)^{m-t}\binom{m}{m-t}{m-(m-t) \brack t}_q. %\qedhere
        \]
    \end{proof}

    \medskip

    Леммы \ref{l:Reduct-BV} и \ref{l:vmt-comb-1} вместе с указанной формулой для
    числа $B(A_{n-1},m)$ завершают доказательство теоремы \ref{th:Enum-St-Id} для
    типа $A_n$.

    \medskip

    Числа $B(\Phi,m)$ исследовались для классических типов в различных ситуациях с
    использованием метода интегрального представления комбинаторных сумм
    (\cite{Egorychev84}, \cite{Eg-Lev-1996}, \cite{sigsam2001} и др.). Эти числа
    указаны явно в выписанном в теореме \ref{th:Enum-St-Id} виде для типов $B_n$ и
    $C_n$ в~\cite[Предложение~17]{reiner97}, а для типа $D_n$ "---
    в~\cite[Теорема~1.2]{reiner04}.

    \medskip

    Учитывая леммы \ref{l:Reduct-BV} и \ref{l:vmt-comb-1}, доказательство теоремы
    \ref{th:Enum-St-Id} завершено.
\end{proof}

\medskip

Решая проблему~1 из \cite{sigsam2001}, теорема \ref{th:Enum-St-Id} завершает
решение задачи~\textbf{(A)} для классических типов. Для исключительных типов ее
решает (Н.\:Д. Ходюня \cite{jsfu2018})

\begin{theorem}\label{th:exhls}
    Число стандартных идеалов алгебры Ли $N\Phi(q)$ исключительного типа
    равно
    \[G_2:\ q+7;\qquad F_4:\ q^4+3\,q^3+44\,q^2+32\,q+25;\]
    %
    \[E_6:\ q^9+3\,q^8+4\,q^7+67\,q^6+69\,q^5+230\,q^4+306\,q^3+94\,q^2+22\,q+37;\]
    %
    \[E_7:\ 2\,(q^{12}+q^{11}+3\,q^{10}+32\,q^9+90\,q^8+118\,q^7+394\,q^6+449\,q^5+{}\]
    \[{}+708\,q^4+300\,q^3-79\,q^2+31\,q+32);\]
    %
    \[E_8:\ q^{16}+3\,q^{15}+4\,q^{14}+7\,q^{13}+237\,q^{12}+239\,q^{11}+693\,q^{10}+1647\,q^9+3554\,q^8+{}\]
    \[{}+4283\,q^7+5829\,q^6+7055\,q^5+3773\,q^4-2361\,q^3-244\,q^2+239\,q+121.\]
\end{theorem}

\section{Специальная каноническая база идеала}\label{sec:canonicalbase}

В связи с проблемой 2 из \cite{sigsam2001} перечисления всех идеалов алгебр Ли
$N\Phi(q)$ классических типов, в \cite[\S\ 4]{vmj2015} рассматривались
специальные канонические базы лиевых идеалов алгебр $NT(n,K)$.

\medskip

Аналогичные базы подалгебр в алгебрах Ли $N\Phi(K)$ выявляет

\begin{lemma}\label{l:LieIdeal}
    Пусть $H$ "--- ненулевая подалгебра алгебры Ли $N\Phi(K)$ над полем
    $K$ и $\mathcal{L} = \mathcal{L}(H)= \{r_1, r_2, \dots, r_m\}$. Тогда
    любую базу пересечения $Q(\mathcal{L})\cap H$ можно дополнить до базы
    в $H$ элементами

    \begin{equation}\label{eq:base1'}
        \alpha_i=  \sum_{j=1}^m a_{ij}e_{r_j}+\alpha_i' ,\quad
        i=1,2,\dots, t,\quad  t= \dim_K\ H / H \cap Q(\mathcal{L}),
    \end{equation}

    где $||a_{ij}||$ "--- $t\times m$-матрица ранга $t$ над $K$\ $(t \leq m)$ и
    $\alpha_i' \in Q(\mathcal{L})$. Кроме того, при фиксированном упорядочении
    $\mathcal{L}$ для однозначно определенных номеров $j_1=1<j_2<\cdots <j_t \leq
        m$ можно считать

    \begin{equation}\label{eq:base-1}
        a_{i,j_i}=1\ (1\leq i\leq t),\quad a_{ik}=0\ (k<j_i),\quad
        a_{i,j_k}=0,\ 1\leq i<k\leq t.
    \end{equation}
\end{lemma}

\begin{proof} Ясно, что $Q(\mathcal{L})+H=Q(\mathcal{L})+\mathcal{F}(H)$
    есть стандартный идеал в $N\Phi(K)$, имеющий вид $H(\mathcal{L},S)$
    из \eqref{hls}. Элементы $\alpha_i\in H$ и равенства
    \eqref{eq:base-1} получаем, используя известный изоморфизм
    \[
        (H+Q(\mathcal{L})) / Q(\mathcal{L}) \simeq H / H \cap Q(\mathcal{L}).
    \]

    Уточним матрицу $||a_{ij}||$ из \eqref{eq:base1'}. Очевидно, базисный элемент
    $\alpha_1\in H$ всегда можно выбрать так, что $a_{11}=1$; положим $j_1=1$ и
    обозначим через $H_1$ подалгебру элементов из $H$ с нулевым коэффициентом при
    $e_{r_1}$. Если $\alpha_i$, $j_i$ и $H_i$ уже определены при $1\leq i<t$, то
    элемент $\alpha_{i+1}\in H_i$ выбираем с наименьшим возможным номером $j_{i+1}$
    первой ненулевой координаты; можно считать $a_{i+1,k}=1$ при $k=j_{i+1}$. Через
    $H_{i+1}$ обозначаем подалгебру элементов из $H_i$ с нулевым коэффициентом при
    $e_{r_k}, \ k=j_{i+1}$.

    Продолжая аналогично, на $t$-м шаге выбираем в $H_{t-1}$ элемент $\alpha_{t}$ с
    коэффициентом $a_{t,j_t}=1$ и $H_{t}=H \cap Q(\mathcal{L})$. Элементарные
    преобразования дают оставшиеся в~\eqref{eq:base-1} равенства $a_{i,j_k}=0$
    последовательно для $i=1,2, \cdots , t-1$ и $i<k\leq t$. Это завершает
    доказательство леммы.
\end{proof}

Уточним базу \eqref{eq:base1'} для нестандартного идеала $H$ обертывающей
алгебры $RD_n(K)$. По теореме \ref{th:classical-Exc-RDn}, $Q(r)\subset H$ для
всех его углов $r$, кроме двух симметричных углов $s$ и $\bar{s}$. Простой
корень $p \neq \bar{p}$, инцидентный с $s$, и простой корень $p_0= \bar{p}_0$
такой, что $p+p_0\in\Phi^+$, определены однозначно, причем $T(s_0 + \bar{p})
    \subset H$, где полагаем $s_0=s+p_0$ при $s=p$ и $s_0=s$ при $s\neq p$.

\medskip

Если неинцидентные с $s$ и $\bar{s}$ простые корни $p_j= \bar{p}_j$ выберем
так, что
%
\begin{equation}\label{eq:r-stepsId}
    s_i=s+p_1+p_2+\cdots + p_i\in \Phi^+, \quad 1\leq i< n-ht(s),
\end{equation}
%
то $Q(s_{i})\subseteq H$ при $i=n-1-ht(s)$. Поэтому существует наименьший номер
$k$ с условиями $Q(s_k) \subseteq H$ и $1\leq k\leq n-1-ht(s)$. Записывая
$\mathcal{L}$ в виде
\begin{equation}\label{eq:R-stepsId}
    \mathcal{L}=\{r_1=s, r_2=\bar{s}, r_3, \dots, r_m\}\quad (2\leq
    m=|\mathcal{L}| <n),
\end{equation}
при $3\leq j\leq m$ имеем $\bar{r}_j= r_j$. Сопоставим $H$, по лемме
\ref{l:LieIdeal}, $t\times m$-матрицу $||a_{ij}||$ над $K$ ранга $t$\ $(1\leq t
    < m)$ с условиями \eqref{eq:base-1}. Тогда $a_{12}=c\neq 0$ и
\[
    H \cap Q(\mathcal{L}) =T(s_0 + \bar{p}) + Q(s_k) + Q(\bar{s}_k)+
    \sum_{j=1}^k K(e_{s_j} + ce_{\bar{s}_j})+ \sum_{j=3}^m Q(r_j)
\]
(последнее слагаемое при $m=2$ отбрасываем). Упрощая в
\eqref{eq:base1'} элементы $\alpha_i$ по модулю $H \cap Q(\mathcal{L})$,
находим $d_i\in K$ такие, что $\alpha_i'= d_ie_{s_k}, \
    i=1,2, \dots, t.$

\medskip

Таким образом, идеал $H$ представляется в виде

\begin{equation}\label{eq:IdH0}
    Id\ \{ \mathcal{L}, k, t, ||a_{ij}||,\ c, d_1, \dots d_t \}\ \coloneq
    \sum_{i=1}^t \ K(d_ie_{s_k}+\sum_{j=1}^m a_{ij}e_{r_j})+
\end{equation}
\[
    +T(s_0 + \bar{p})+ Q(s_k)+ Q(\bar{s}_k)+
    \sum_{j=1}^k K(e_{s_j} + ce_{\bar{s}_j})+\sum_{j=3}^m Q(r_j),
\]
где корни $s_i$ определены в \eqref{eq:r-stepsId}. Его однозначно характеризуют
множество углов $\mathcal{L}=\mathcal{L}(H)$ вида \eqref{eq:R-stepsId},
параметры $k, t$ с условиями
\[
    1\leq k\leq n-1-ht(s),\quad 1\leq t= \dim_K\ H / H \cap Q(\mathcal{L}) <|\mathcal{L}|=m <n,
\]
элементы $c\neq 0$, $d_1, \dots , d_t$ из $K$ и $t\times m$-матрица
$||a_{ij}||$ над $K$ ранга $t$\ с условиями $a_{12}=c$ и \eqref{eq:base-1}. Тем
самым, доказана

\begin{theorem}\label{th:ExIdForm}
    В обертывающей алгебре $RD_n(K)$ всякий нестандартный идеал $H$
    имеет множество углов $\mathcal{L}=\mathcal{L}(H)$ вида
    \eqref{eq:R-stepsId} и представляется как идеал \eqref{eq:IdH0},
    однозначно характеризуемый набором $\mathcal{L}, k, t, ||a_{ij}||,\
        c, d_1, \dots d_t$.
\end{theorem}

\section{Метод коэффициентов и перечисление идеалов алгебры $RD_n(q)$}\label{sec:egorychev-RDn}
Теорема \ref{th:Enum-St-Id}, решая проблему 1 из \cite{sigsam2001} о числе всех
стандартных идеалов алгебры Ли $N\Phi(q)$ классического типа над полем
$K=GF(q)$, дает и число всех идеалов ее обертывающей алгебры $R_{\Phi}(q)$,
когда она стандартна.

В силу теорем \ref{th:Enum-St-Id} и \ref{th:classical-Exc-RDn}, перечисление
идеалов любой стандартной обёртывающей алгебры из
предложения~\ref{p:mult-enveloping} классических типов $A_n$, $B_n$ и $C_n$
завершено. Остается перечислить нестандартные идеалы обёртывающей алгебры
$RD_n(q)$.

\medskip

Основной в этом параграфе является

\begin{theorem}\label{th:Enum-Dn}
    Число нестандартных идеалов алгебры $RD_n(q)$ равно
    \[
        \sum_{m=2}^{n-1} \binom{n-2}{m-2}\binom{n-1}{m}
        \sum_{t=1}^{m-1}q^t(q-1)
        \sum_{j=0}^{m-1-t} (-1)^j \binom{m-1}{j} {m-1-j \brack t}_q.
    \]
\end{theorem}

\begin{proof}
    Нестандартные идеалы алгебры $RD_n(q)$, по теореме~\ref{th:ExIdForm},
    исчерпываются идеалами вида~\eqref{eq:IdH0},
    которые характеризуются однозначно набором $\mathcal{L}, k, t,
        ||a_{ij}||,\ c, d_1, \dots d_t$. Записывая $\mathcal{L}$ как в
    \eqref{eq:R-stepsId} в матричном представлении, считаем
    \[
        e_{s}=e_{i1},\quad e_{\bar{s}}=e_{i,-1},\quad e_{s_k}=e_{i+k,1}\quad (2\leq i< n-k).
    \]
    Через $B(m,n,i,k)$ обозначим число всевозможных $(m+2)$-элементных множеств
    углов вида \eqref{eq:R-stepsId}, у которых два угла соответствуют базисным
    элементам $e_{s}$ и $e_{\bar{s}}$ и нет углов в строках с номерами $j$, $i < j
        \leq k$. Нам потребуется

    \begin{lemma}\label{lemma:rdn}
        Число нестандартных идеалов алгебры $RD_n(q)$ равно
        \begin{equation}\label{eq:enum-Dn-1}
            \sum_{m=2}^{n-1} \sum_{i=2}^{n-1} \sum_{k=1}^{n-i}
            B(m,n,i,k)\sum_{t=1}^{m-1}q^t(q-1)\widetilde{V}(m-1,t,q).
        \end{equation}
    \end{lemma}

    \begin{proof}
        При фиксированных $|\mathcal{L}|= m$ и $k$ число способов выбора
        параметра $c\neq 0$ в $K=GF(q)$ и наборов $(d_1, \dots , d_t)$
        над $K$ равно $q^t(q-1)$. Согласно доказательству леммы
        \ref{l:LieIdeal}, число возможностей выбора матриц $||a_{ij}||$,
        требуемых для канонических базисов идеалов, совпадает с
        $\widetilde{V}(m-1,t,q)$ при любых $t,m$. Отсюда для числа
        нестандартных идеалов алгебры $RD_n(q)$ получаем формулу
        \eqref{eq:enum-Dn-1}.
    \end{proof}

    Второе и третье суммирования в комбинаторной сумме~\eqref{eq:enum-Dn-1}
    проведем с помощью метода коэффициентов. Напомним некоторые понятия.

    Пусть $L$ "--- множество формальных степенных рядов Лорана над полем
    $\mathbb{C}$, содержащих конечное число членов с отрицательными степенями. По
    определению, целое число $k$ есть \emph{порядок} монома $c_kw^k\ \neq 0$ и
    $L_k$ "--- множество рядов Лорана из $L$, у которых наименьший порядок мономов
    равен $k$. Под \emph{оператором формального вычета} ряда $C(w) =
        \sum_{k}c_kw^k$ из $L$ полагают
    \[
        \res_wC(w) \coloneq c_{-1}.
    \]

    Оператор формального вычета $\res$ (контурный интеграл) имеет ряд свойств (см.,
    например, \cite{Egorychev84}). Из них наиболее часто используются следующие
    два.

    \medskip

    \textbf{Правило линейности.} Для любых $A(w)$ и $B(w)$ из $L$ и
    $a,b \in \mathbb{C}$ имеем
    \[
        a\res_wA(w) + b\res_wB(w) = \res_w\{aA(w) + bB(w)\}.
    \]

    \textbf{Правило подстановки.} Если либо $f \in L_m \
        (m=1,2,\dots)$ и $A(w) \in L$, либо $A(w)$ "--- полином и $f(w)\in
        L$, то
    %
    \[\sum_k (f(w))^k\res_z\{A(z)z^{-k-1}\} = A(f(w)).\]

    Если $C(w)=\sum_kc_kw^k$ из $L$ есть производящая функция для
    последовательности $\{c_k\}$, то $c_k=\res_wC(w)w^{-k-1}$ для любого $k$. Когда
    $n,k = 0,1,2,\dots$, приходим к биномиальным коэффициентам
    %
    \begin{equation}\label{integral_formula}
        \binom{n}{k} = \res_w\{\frac{(1+w)^n}{w^{k+1}}\} = \frac{1}{2\pi
            i}\int_{\lvert
            w \rvert=\theta} \frac{(1+w)^n}{w^{k+1}}\,dw,\ \theta > 0,
    \end{equation}
    \[
        \binom{-n}{k} \coloneq \binom{n+k-1}{k} =
        \res_w\{\frac{(1-w)^{-n}}{w^{k+1}}\} = \frac{1}{2\pi
            i}\int_{\lvert
            w \rvert=\theta} \frac{(1-w)^{-n}}{w^{k+1}}\,dw,\ 0 < \theta < 1.
    \]

    \medskip

    В доказательстве теоремы~\ref{th:Enum-Dn} основной является
    %
    \begin{lemma}\label{B'-Dn}
        Справедлива следующая формула суммирования
        \[\sum_{i=2}^{n-1} \sum_{k=1}^{n-i} B(m,n,i,k) =
            \binom{n-2}{m-2}\binom{n-1}{m}.\]
    \end{lemma}

    \begin{proof}
        Обозначим через $B^{(-)}(u,l)$ число $l$-элементных множеств углов
        в $D_n^+$-матрице таких, что последний угол расположен левее
        $(-1)$-го столбца и выше $(u+3)$-й строки, а через
        $B^{(+)}(u,n-u-1,l)$ "--- число $l$-элементных множеств углов таких,
        что первый угол расположен правее $1$-го столбца и ниже $(n-u)$-й
        строки. В этих обозначениях
        %
        \begin{equation}\label{Bmnik}
            B(m,n,i,k)= \sum_{l=0}^m B^{(-)}(i-3,l)
            B^{(+)}(n-i-k,i+k-1,m-2-l).
        \end{equation}
        %
        Применяя \cite[Теорема~10.14.1]{Krattenthaler2015}, находим

        \begin{equation}
            B^{(+)}(u,v,l) =
            \binom{u+v}{l}\binom{u}{l} - \binom{u+v-1}{l-1}\binom{u+1}{l+1}.
        \end{equation}
        Несложная индукция по $v$ дает также равенство
        \begin{eqnarray}\label{bminus}
            B^{(-)}(u,v) = \binom{u+1}{2v}.
        \end{eqnarray}

        В силу~\eqref{Bmnik}-\eqref{bminus}, второе и третье суммирования
        в~\eqref{eq:enum-Dn-1} приводят к 3-кратной комбинаторной сумме с биномиальными
        коэффициентами. При этом, как обычно, вычисления проводим с помощью свойств
        оператора $\res$, путем последовательного нахождения интегрального
        представления рассматриваемых выражений и вычисления полученных интегралов. В
        соответствии с \cite[теорема 10.14.1]{Krattenthaler2015} находим представление
        чисел $B^+(u,v,l)$. Пользуясь интегральной формулой~\eqref{integral_formula},
        для каждого из биномиальных коэффициентов находим
        \begin{gather*}
            B^{(+)}(u,v,l) =
            \binom{u+v}{l}\binom{u}{l} - \binom{u+v-1}{l-1}\binom{u+1}{l+1} = \\
            = \res_{z,w}\left\{ \frac{(1+z)^{u+v}}{z^{l+1}}\frac{(1+w)^u}{w^{l+1}}
            \right\}
            - \res_{z,w}\left\{ \frac{(1+z)^{u+v-1}}{z^l}\frac{(1+w)^{u+1}}{w^{(l+1)+1}}
            \right\} = \\
            = \res_{z,w}\left\{ \frac{(1+z)^{u+v-1}(1+w)^u}{z^{l+2}}(w-z) \right\}
        \end{gather*}
        и, тем самым,
        \begin{equation}\label{integral_formula_bplus}
            \begin{gathered}
                B^{(+)}(n-i-k, i+k-1, m - 2 - l) \coloneq \\
                = \res_{z,w}\left\{ \frac{(1+z)^{n-2}(1+w)^{n-i-k}}{z^{(m-2-l) +
                            1}w^{(m-1-l)
                            + 1}}(w-z) \right\}.
            \end{gathered}
        \end{equation}

        Отсюда, в силу \eqref{Bmnik}-\eqref{bminus}, получаем
        \begin{gather*}
            S \coloneq \sum_{i=2}^{n-1}\sum_{k=1}^{n-i}B(m,n,i,k) = \\
            = \sum_{i=2}^{n-1}\sum_{k=1}^{n-i}\sum_{l=0}^{m}B^{(-)}(i-3, l)
            B^{(+)}(n-i-k, i+k-1, m-2-l) =
        \end{gather*}
        (занесение знака суммы за знак $\res_{z,w}$ и добавление нулевых
        слагаемых)
        \begin{gather*}
            \sum_{i=2}^{n-1}\sum_{k=1}^{n-i}\res_{z,w}\left\{
            \frac{(1+z)^{n-2}(1+w)^{n-i-k}(w-z)}{z^{(m-2)+1}w^{(m-1)+1}}
            \times \left[ \sum_{l=0}^{m}(zw)^l \binom{i-2}{2l} \right]\right\}=
        \end{gather*}
        (суммирование в квадратных скобках и формула геометрической
        прогрессии)
        \begin{gather*}
            \sum_{i=2}^{n-1}\sum_{k=1}^{n-i}\res_{z,w}\left\{
            \frac{(1+z)^{n-2}(1+w)^{n-i-k}(w-z)}{z^{(m-2)+1}w^{(m-1)+1}}
            \times \frac{(1 + \sqrt{zw})^{i-2} + (1 - \sqrt{zw})^{i-2}}{2}
            \right\}
            = \\
            \frac{1}{2}\res_{z,w} \{
            %
            \frac{(1+z)^{n-2}(1+w)^{n-2}(w-z)}{z^{(m-2)+1}w^{(m-1)+1}} \times \\
            \left[ \sum_{i=2}^{\infty} \left(
                \frac{1+\sqrt{zw}}{1+w}\right)^{i-2} + \sum_{i=2}^{\infty} \left(
                \frac{1+\sqrt{zw}}{1+w}\right)^{i-2} \right] \times
            \sum_{k=1}^{\infty}\left(\frac{1}{1+w}\right)^k
            %
            \} =
        \end{gather*}
        ($\lvert w\rvert \gg 1$, $\lvert (1 + \sqrt{zw})/(1+w)\rvert < 1$,
        $\lvert (1-\sqrt{zw})/(1+w) \rvert < 1$, суммирование по индексам
        $i$ и $k$, формула геометрической прогрессии)
        \begin{gather*}
            \frac{1}{2}\res_{z,w}\{
            \frac{(1+z)^{n-2}(1+w)^{n-2}(w-z)}{z^{(m-2)+1}w^{(m-1)+1}} \times \\
            \left[ 1/(1 - \frac{1+ \sqrt{zw}}{1+w}) + 1 / (1 - \frac{1 -
                    \sqrt{zw}}{1+w}) \right] \times \frac{1}{1+w} \times \frac{1}{1 -
                \frac{1}{1+w}}
            \} = \\
            \frac{1}{2}\res_{z,w}\left\{
            \frac{(1+z)^{n-2}(1+w)^{n-1}(w-z)}{z^{(m-2)+1}w^{(m-1)+1}} \times
            \left[ \frac{1}{w-\sqrt{zw}} + \frac{1}{w + \sqrt{zw}} \right]
            \times \frac{1}{w}
            \right\} = \\
            \frac{1}{2}\res_{z,w}\left\{
            \frac{(1+z)^{n-2}(1+w)^{n-1}(w-z)}{z^{(m-2)+1}w^{(m-1)+1}} \times
            \frac{2w}{w^2-zw} \times \frac{1}{w}
            \right\} = \\
            \res_{z,w}\left\{
            \frac{(1+z)^{n-2}(1+w)^{n-1}}{z^{(m-2)+1}w^{m+1}} \right\} =
            \\
            \res_{z,w}\left\{ \frac{(1+z)^{n-2}}{z^{(m-2)+1}} \right\} \times
            \res_{z,w}\left\{ \frac{(1+w)^{n-1}}{w^{m+1}} \right\} \coloneq
            \binom{n-2}{m-2}\binom{n-1}{m}.
        \end{gather*}
        Тем самым, доказательство леммы \ref{B'-Dn} завершается.
    \end{proof}

    Применяя к лемме \ref{lemma:rdn} леммы \ref{l:vmt-comb-1} и \ref{B'-Dn}
    получаем утверждение теоремы \ref{th:Enum-Dn}.
\end{proof}

\begin{remark}\label{remark:problem-2}
    В отличие от идеалов алгебры $RD_n(K)$, для идеалов $H$ алгебры Ли
    $N\Phi(K)$ классического типа возрастает число пар $p$-связанных
    углов $r,s$ в $H$.
    Решение проблемы 2 из \cite{sigsam2001} сложнее даже
    для типа $A_n$; с возрастанием ранга $\Phi$
    может неограниченно возрастать и число углов, $p$-связанных не с
    одним, а с двумя углами в $H$.
\end{remark}

\FloatBarrier

\printnomenclature[3.5cm] % Значение ширины столбца с обозначениями стоит подбирать вручную
\include{Dissertation/references}      % Список литературы

\setcounter{totalchapter}{\value{chapter}} % Подсчёт количества глав

% Оформление заголовков приложений ближе к ГОСТ:
\setlength{\midchapskip}{20pt}
\renewcommand*{\afterchapternum}{\par\nobreak\vskip \midchapskip}

\end{document}
